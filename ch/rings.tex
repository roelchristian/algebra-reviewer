\chapter{Rings}
\label{ch:rings}

\section{Rings and homomorphisms}

\begin{definition}
    A \emph{ring} is a nonempty set \(R\) together with two binary operations
    \(+\) and \(\cdot\) (often called `addition' and `multiplication') such
    that:
    \begin{enumerate}[label=(\alph*)]
        \item \((R, +)\) is an abelian group with identity element \(0\),
        \item Multiplication is associative, i.e. \((a \cdot b) \cdot c = a
        \cdot (b \cdot c)\) for all \(a, b, c \in R\),
        \item Multiplication is distributive over addition, i.e. \(a \cdot (b +
        c) = a \cdot b + a \cdot c\) and \((a + b) \cdot c = a \cdot c + b \cdot
        c\) for all \(a, b, c \in R\).
    \end{enumerate}

    If, in addition, the ring has a multiplicative identity element \(1\) such
    that \(1 \cdot a = a \cdot 1 = a\) for all \(a \in R\), then we say that
    \(R\) is a \emph{ring with unity} or a \emph{ring with \(1\)}.

    If multiplication is commutative, i.e. \(a \cdot b = b \cdot a\) for all
    \(a, b \in R\), then we say that \(R\) is a \emph{commutative ring}.
\end{definition}

\begin{remark}
    When we speak of a ring with unity, we will often assume without explicitly
    stating that the multiplicative identity \(1\) is distinct from the additive
    identity \(0\). In case \(1 = 0\) we get the trivial ring \(\{0\}\) (also
    called the \emph{zero ring}) whose only element is the additive and
    multiplicative identity.
\end{remark}

\begin{remark}
    As in Chapter~\ref{ch:groups}, we will often write \(ab\) instead of \(a
    \cdot b\).
\end{remark}

\begin{example}
    The sets \(\Z\), \(\Q\), \(\R\), and \(\C\) are all rings with unity under
    the usual operations of addition and multiplication.
\end{example}

\begin{example}
    The set of \(n \times n\) matrices with entries in a ring \(R\) is a ring
    with unity under matrix addition and multiplication, denoted
    \(\mathcal{M}_n(R)\). The ring \(\mathcal{M}_n(R)\) is commutative if and
    only if \(R\) is commutative and \(n = 1\).
\end{example}

\begin{theorem}
    Let \(R\) be a ring. Then
    \begin{enumerate}[label=(\alph*)]
        \item \(0a = a0 = 0\) for all \(a \in R\);
        \item \((-a)b = a(-b) = -(ab)\) for all \(a, b \in R\);
        \item \((-a)(-b) = ab\) for all \(a, b \in R\);
        \item \((na)b = n(ab) = a(nb)\) for all \(a, b \in R\) and \(n \in
        \mathbb{Z}\);
        \item For all \(a, b \in R\),
        \[
            \left(\sum_{i=1}^{n} a_i\right)\left(\sum_{j=1}^{m} b_j\right) = \sum_{i=1}^{n} \sum_{j=1}^{m} a_i b_j.
        \]
    \end{enumerate}
\end{theorem}

\begin{proof}
    (a) \((0)a = (0 + 0)a = 0a + 0a\), whence \(0a = 0\). (b) \((-a)b = (-a)b +
    ab = ((-a) + a)b = 0b = 0\). This implies that \((-a)b = -(ab)\). A similar
    argument shows that \(a(-b) = -(ab)\). (c) \((-a)(-b) + ab = (-a)(-b) +
    (-a)b = (-a)(-b + b) = (-a)0 = 0\). (d) We prove this by induction on \(n\).
    The case \(n = 0\) is trivial. Suppose the result holds for \(n = k\). Then
    \[
        ((k + 1)a)b = (ka + a)b = kab + ab = (k+1)(ab)
    \]
    and
    \[
        a((k + 1)b) = a(kb + b) = k(ab) + ab = (k+1)(ab).
    \]
    Statement (e) is proved similarly.
\end{proof}

\begin{theorem}
    In a ring with unity, the element \(1\) is unique.
\end{theorem}

\begin{proof}
    Suppose that \(1\) and \(1'\) are both multiplicative identities in a ring
    \(R\). Then \(1 = 1 \cdot 1' = 1'\).
\end{proof}

\begin{example}
    Show that the set \(\{0, 2, 4, 6, 8\}\) with addition and multiplication
    modulo \(10\) is a ring with unity.

    \begin{solution}
        We have the following addition and multiplication tables:
        \begin{center}
            \begin{tabular}{c|ccccc}
                \(+\) & 0 & 2 & 4 & 6 & 8 \\
                \hline
                0 & 0 & 2 & 4 & 6 & 8 \\
                2 & 2 & 4 & 6 & 8 & 0 \\
                4 & 4 & 6 & 8 & 0 & 2 \\
                6 & 6 & 8 & 0 & 2 & 4 \\
                8 & 8 & 0 & 2 & 4 & 6
            \end{tabular}
            \quad
            \begin{tabular}{c|ccccc}
                \(\cdot\) & 0 & 2 & 4 & 6 & 8 \\
                \hline
                0 & 0 & 0 & 0 & 0 & 0 \\
                2 & 0 & 4 & 8 & 2 & 6 \\
                4 & 0 & 8 & 6 & 4 & 2 \\
                6 & 0 & 2 & 4 & 6 & 8 \\
                8 & 0 & 6 & 2 & 8 & 4
            \end{tabular}
        \end{center}

        We see that the set is closed under addition and multiplication. The
        identity element for addition is \(0\) and for multiplication \(6\). The
        ring is commutative and has a unity.
    \end{solution}
\end{example}

\begin{definition}
    A nonzero element \(a\) in a ring \(R\) is called a \emph{left} (resp.
    \emph{right}) \emph{zero divisor }if there exists a nonzero element \(b \in
    R\) such that \(ab = 0\) (resp. \(ba = 0\)). An element that is both a left
    and right zero divisor is called a \emph{zero divisor.}
\end{definition}

\begin{definition}
    An element \(a\) in a ring \(R\) with unity is said to be \emph{left} (resp.
    \emph{right}) \emph{invertible} if there exists an element \(b \in R\) such
    that \(ba = 1\) (resp. \(ab = 1\)). An element that is both left and right
    invertible is called \emph{invertible} or a \emph{unit.}
\end{definition}

\begin{definition}
    A commutative ring with unity is called an \emph{integral domain} (or simply
    a \emph{domain}) if it has no zero divisors. A ring with unity is called a
    \emph{division ring} if every nonzero element is a unit. A commutative
    division ring is called a \emph{field}.
\end{definition}

\begin{theorem}
    The cancellation laws hold in an integral domain. That is for all \(a, b, c
    \in R\), if \(ab = ac\) and \(a \neq 0\), then \(b = c\).
\end{theorem}

\begin{proof}
    From \(ab = ac\), we have \(ab - ac = 0\), which implies \(a(b - c) = 0\).
    Since \(a \neq 0\), we must have \(b - c = 0\), i.e. \(b = c\).
\end{proof}

\begin{theorem}
    A finite integral domain is a field.
    \label{thm:finite-integral-domain-field}
\end{theorem}

\begin{proof}
    Let \(D\) be a finite integral domain and \(a\) an element of \(D\) such
    that \(a \neq 0\). Consider the set \(\{a^n \mid n \in \mathbb{N}\}\). If
    \(a = 1\) then \(a\) is a unit and so we may assume that \(a \neq 1\). Since
    \(D\) is finite, there exist positive integers \(j\) and \(k\) with \(j <
    k\) such that \(a^j = a^k\). Because \(D\) is an integral domain, we have
    \(a^{j-k} = 1\) by cancellation. Now since \(a \neq 1\), we have \(j - k >
    1\), which implies that \(a\) is a unit with inverse \(a^{j-k-1}\). Since
    \(a\) was arbitrary, we conclude that every nonzero element of \(D\) is a
    unit.
\end{proof}

\begin{theorem}
    For every prime \(p\) the ring \(\mathbb{Z}/p\mathbb{Z}\) is a field.
\end{theorem}

\begin{proof}
    By Theorem~\ref{thm:finite-integral-domain-field}, it suffices to show that
    \(\mathbb{Z}/p\mathbb{Z}\) is an integral domain. Now suppose that for some
    \(a, b \in \mathbb{Z}/p\mathbb{Z}\) we have \(ab = 0\). Then \(ab = pk\) for
    some integer \(k\), which implies that \(p \mid ab\). Since \(p\) is prime,
    we must have \(p \mid a\) or \(p \mid b\), which implies that \(a = 0\) or
    \(b = 0\). Thus \(\mathbb{Z}/p\mathbb{Z}\) is an integral domain.
\end{proof}

\begin{theorem}[Binomial theorem]
    Let \(R\) be a commutative ring with unity, \(n\) a positive integer, and
    \(a, b \in R\). Then
    \[
        (a + b)^n = \sum_{k=0}^{n} \binom{n}{k} a^{n-k} b^k.
    \]
\end{theorem}

\begin{definition}
    Let \(R\) and \(S\) be rings. A function \(\phi: R \to S\) is called a
    \emph{ring homomorphism} if it preserves addition and multiplication, i.e.
    for all \(a, b \in R\),
    \[
        \phi(a + b) = \phi(a) + \phi(b) \quad \text{and} \quad \phi(ab) = \phi(a)\phi(b).
    \]

    The same terminology we have introduced in Chapter~\ref{ch:groups},
    \S~\ref{def:homomorphism-group} for homomorphism of groups will be used for
    homomorphisms of rings.
\end{definition}

\begin{definition}
    Let \(R\) be a ring. If there is a least positive integer \(n\) such that
    \(na = 0\) all \(a \in R\), then \(n\) is called the \emph{characteristic}
    of \(R\), written \(\charx R\). If no such integer exists, then we say that
    the characteristic of \(R\) is zero.
\end{definition}

\section{Ideals}

\begin{definition}
    Let \(R\) be a ring and \(S\) a nonempty subset of \(R\) that is closed
    under addition and multiplication by elements of \(R\). Then \(S\) is called
    a \emph{subring} of \(R\) if it is a ring with respect to the operations of
    \(R\).
\end{definition}

\begin{definition}
    A subring \(I\) of a ring \(R\) is called an \emph{left} (resp.
    \emph{right}) \emph{ideal} of \(R\) if for all \(a \in I\) and \(r \in R\),
    we have \(ra \in I\) (resp. \(ar \in I\)). If \(I\) is both a left and right
    ideal, then we say that \(I\) is an \emph{ideal} of \(R\).
\end{definition}

\begin{example}
    For any ring \(R\), \(R\) itself and the zero ring \(\{0\}\) are ideals of
    \(R\). If \(I\) is neither \(R\) nor \(\{0\}\), then \(I\) is called a
    \emph{proper ideal} of \(R\).
\end{example}

\begin{remark}
    \label{rem:ideal-contains-0}
    Every ideal of a ring \(R\) contains the additive identity \(0\). This
    follows from the fact that for all \(a \in I\), we have \(0a = 0 \in I\).
\end{remark}

\begin{theorem}[Test for ideals]
    A nonempty subset \(I\) of a ring \(R\) is an left (resp. right) ideal of
    \(R\) if and only if for all \(a, b \in I\) and \(r \in R\), we have \(a - b
    \in I\) and \(ra \in I\) (resp. \(ar \in I\)).
\end{theorem}

\begin{proof}
    We prove the test for a left ideal. To prove necessity, suppose that \(I\)
    is an ideal. Then the statements \(a - b \in I\) and \(ra \in I\) follow
    directly from the ring axioms and the definition of a left ideal. On the
    other hand, suppose that \(I\) is a nonempty subset of \(R\) such that \(a -
    b \in I\) and \(ra \in I\) for all \(a, b \in I\) and \(r \in R\). Since
    \(I\) is nonempty, there exists an element \(a \in I\) and from our
    assumption we have \(a - a = 0 \in I\). This further implies that \(-a \in
    I\) for all \(a \in I\). From this it follows that \(a + b \in I\) and \(I\)
    is closed under addition. Now since \(b \in I\) implies \(b \in R\), it
    follows that \(ba \in I\). Thus \(I\) is closed under addition and
    multiplication (as defined in \(R\)) and hence \(I\) is a subring. Since
    \(I\) is a subring such that \(ra \in I\) for all \(a \in I\) and \(r \in
    R\), it follows from the definition that \(I\) is a left ideal.
\end{proof}

\begin{theorem}
    Let \(R\) be a commutative ring with unity and \(I\) an ideal of \(R\). If a
    unit \(u\) of \(R\) is in \(I\), then \(I = R\).
\end{theorem}

\begin{proof}
    Let \(v \in R\) be the inverse of \(u\). From the definition of an ideal, we
    have \(uv = 1 \in I\) but this would imply that \(r = 1r \in I\) for all \(r
    \in R\) and thus \(R \subseteq I\). Sine \(I \subseteq R\), we have \(I =
    R\).
\end{proof}

\begin{theorem}
    Let \(\mathcal{I}\) be a nonempty collection of left (resp. right) ideals of
    a ring \(R\). Then the intersection of all the ideals in \(\mathcal{I}\) is
    an left (resp. right) ideal of \(R\).
\end{theorem}

\begin{proof}
    From Remark~\ref{rem:ideal-contains-0}, each \(I \in \mathcal{I}\) contains
    the additive identity \(0\). Thus the intersection of all the ideals in
    \(\mathcal{I}\) is nonempty and contains \(0\). Let \(a, b\) be elements of
    the intersection and \(r\) an element of \(R\). Then for all \(I \in
    \mathcal{I}\), we have \(a, b \in I\) and hence \(a - b \in I\). This
    implies that \(a - b\) is in the intersection. Similarly, for all \(I \in
    \mathcal{I}\), we have \(ra \in I\) and hence \(ra\) is in the intersection.
    Thus the intersection of all the ideals in \(\mathcal{I}\) is a left ideal.
\end{proof}

\begin{definition}
    Let \(X\) be a nonempty subset of a ring \(R\). The \emph{ideal generated
    by} \(X\), denoted \((X)\), is the intersection of all ideals of \(R\) that
    contain \(X\). The elements of \(X\) are called \emph{generators} of the
    ideal \((X)\). If \(X\) is finite, we say that \((X)\) is a \emph{finitely
    generated ideal}.

    If \(X\) is a singleton \(\{a\}\), then we write \((a)\) instead of
    \((\{a\})\). We call such an ideal a \emph{principal ideal}. A
    \emph{principal ideal ring} is a ring in which every ideal is principal. A
    principal ideal ring which is also an integral domain is called a
    \emph{principal ideal domain} (PID).
\end{definition}

\begin{theorem}
    Let \(\phi: R \to S\) be a ring homomorphism. Then the kernel of \(\phi\) is
    an ideal of \(R\).
\end{theorem}

\begin{proof}
    
\end{proof}