\chapter{Groups}
\label{ch:groups}

\section{Definition and basic properties}

\begin{definition}
    A \emph{group} is a pair \((G, \mult)\) where \(G\) is a  set and \(\mult\)
    is a binary operation on \(G\) (i.e., a map from \(G \times G\) to \(G\))
    that satisfies the following axioms:
    \begin{enumerate}
        \item \emph{Associativity:} For all \(a, b, c \in G\), \(\mult(\mult(a,
        b), c) = \mult(a, \mult(b, c))\).
        \item \emph{Identity element:} There exists an element \(e \in G\) such
        that for all \(a \in G\), \(\mult(e, a) = \mult(a, e) = a\). This
        element is called the \emph{identity element} (or the \emph{neutral
        element}) of the group.
        \item \emph{Inverses:} For each \(a \in G\), there exists an element
        \(a^{-1} \in G\) such that \(\mult(a, a^{-1}) = \mult(a^{-1}, a) = e\).
    \end{enumerate}
\end{definition}

If the pair \((G, \mult)\) satisfies only the first two axioms, then it is
called a \emph{monoid}. If it satisfies only the first axiom, then it is called
a \emph{semigroup}.

We shall succumb to some abuse of language and refer to the group \((G, \mult)\)
simply as the group \(G\) when the binary operation \(\mult\) is clear from the
context. We shall also write \(ab\) or \(a \cdot b\) instead of \(\mult(a, b)\)
for the image of \((a, b)\) under the binary operation. We shall call the binary
operation `multiplication' (or more generally as the `group operation' of the
group \(G\)) and the element \(ab\) the `product' of \(a\) and \(b\). The set
\(G\) is called the `underlying set' of the group \(G\). When talking about two
groups \(G\) and \(H\), we shall often write as a subscript on \(\mult\) and
\(e\) the underlying set of the group in order to specify in which group we are
defining those terms.

Because we require the existence of an identity element in a group \(G\), we can
conclude that the set \(G\) is nonempty. At the very least it must contain the
identity element. This group is called the \emph{trivial group} and is denoted
by \(\{e\}\). Note also that any singleton \(\{*\}\) is a group if we define the
binary operation to be the trivial operation \(\mult(*, *) = *\) so that the
choice of writing \(\{e\}\) instead of \(\{*\}\) or any other symbol is purely a
matter of convention.

\begin{theorem}
    The identity element in a group is unique. For each \(a \in G\), the inverse
    element of \(a\) is unique.
\end{theorem}

\begin{proof}
    Suppose \(e\) and \(e'\) are both identity elements of a group \(G\). Then
    \[
        e = e \cdot e' = e'.
    \]
    Similarly, suppose \(b\) and \(b'\) are both inverses of \(a \in G\). Then
    \[
        b = b \cdot e = b \cdot (a \cdot b') = (b \cdot a) \cdot b' = e \cdot b' = b'.
    \]
\end{proof}


The preceding result allow us to speak of \emph{the} identity element and
\emph{the} inverse of an element in a group.

\begin{theorem}
    Let \(G\) be a group and let \(a \in G\). Then \((a^{-1})^{-1} = a\). If in
    addition, \(b \in G\), then \((ab)^{-1} = b^{-1}a^{-1}\).
\end{theorem}

\begin{proof}
    Suppose \(b \neq a\) is the inverse of \(a^{-1}\) where \(b \in G\). Then
    \(a^{-1}b = e = ba^{-1}\). But \(a\) also satisfies this property, and since
    inverses are unique, we must have \(b = a\).

    On the other hand we have
    \[
        (ab)(b^{-1}a^{-1}) = a(bb^{-1})a^{-1} = aea^{-1} = aa^{-1} = e.
    \]
\end{proof}


\begin{theorem}[Cancellation laws]
    Let \(G\) be a group. For all \(a, b, c \in G\),
    \begin{enumerate}
        \item If \(ab = ac\), then \(b = c\).
        \item If \(ba = ca\), then \(b = c\).
    \end{enumerate}
\end{theorem}

\begin{proof}
    We prove the first statement. Suppose \(ab = ac\). Then
    \begin{align*}
        a^{-1}(ab) &= a^{-1}(ac) &\\
        (a^{-1}a)b &= (a^{-1}a)c &\text{(associativity)}\\
        eb &= ec &\text{(definition of inverse)}\\
        b &= c. &\text{(definition of identity)}
    \end{align*}
    The second statement is proved similarly.
\end{proof}

\begin{corollary}
    For all \(a, b \in G\), the equations \(ax = b\) and \(ya = b\) have unique
    solutions in \(G\).
\end{corollary}

\begin{proof}
    Observe that the solutions to the equations are \(x = a^{-1}b\) and \(y =
    ba^{-1}\), respectively. In the first case we have
    \[
        ax = a(a^{-1}b) = (aa^{-1})b = eb = b.
    \]
    If \(x'\) is also a solution for the first equation then \(xa = b = x'a\)
    implies \(x = x'\) by right cancellation. The second equation is proved
    similarly.
\end{proof}

\begin{theorem}[Generalized associative law]
    \label{thm:generalized-associative-law}
    Let \(G\) be a group and let \(a_1, a_2, \ldots, a_n\) be elements of \(G\).
    Then the product \(a_1 a_2 \cdots a_n\) is well-defined, i.e., it is
    independent of the way in which the product is bracketed.
\end{theorem}

\begin{proof}
    We prove this by induction on \(n\). The case \(n = 2\) is the associative
    axiom in our definition of a group. If \(n > 2\), then from the definition
    of associativity
    \[
        a_1 a_2 \cdots a_n = (a_1 a_2 \cdots a_m)(a_{m + 1} \cdots a_n)
    \]
    for some \(m < n\). By induction we then have
    \begin{align*}
        a_1 a_2 \cdots a_n & = \left(\prod_{i = 1}^m a_i\right)\left(\prod_{i = 1}^{n - m} a_{m + i}\right)\\
        & = \left(\prod_{i =1}^{m}\right)\left(\left(\prod_{i = 1}^{n - m - 1} a_{m + i}\right)a_n\right)\\
        & = \left(\left(\prod_{i = 1}^{m}\right)\left(\prod_{i = 1}^{n - m- 1} a_{m + i}\right)\right)a_n\\
        & = \left(\prod_{i = 1}^{n - 1} a_i\right)a_n.
    \end{align*}
\end{proof}

\begin{remark}
    Theorem \ref{thm:generalized-associative-law} tells us that we can write
    \(a_1 a_2 \cdots a_n\) as a meaningful product without the need for
    brackets. Where each of the \(a_i\) is equal to an element \(a\) of the
    group, we shall write \(a^n\) for the product of \(n\) terms of \(a\). Thus
    \(a^1 = a\), \(a^2 = aa\), and so on. We also define \(a^0\) to be the
    identity element of the group and \(a^{-n}\) to be \((a^{-1})^n\).
\end{remark}

\begin{theorem}
    If \(G\) is a group and \(a \in G\), then for all integers \(m, n\),
    \begin{enumerate}[label=(\alph*)]
        \item \((a^n)^{-1} = (a^{-1})^n\),
        \item \(a^m a^n = a^{m + n}\), and
        \item \((a^m)^n = a^{mn}\).
    \end{enumerate}
\end{theorem}

\begin{proof}
    We prove each part in turn.
    \begin{enumerate}[label=(\alph*), wide]
        \item We prove this by induction on \(n\). The case \(n = 1\) is
        trivial. Suppose the result holds for \(n = k\). Then
        \begin{align*}
            a^{k + 1} & = a^k a\\
            (a^{k + 1})^{-1} & = (a^k a)^{-1} = a^{-1} (a^k)^{-1}\\
            & = a^{-1} (a^{-1})^k = (a^{-1})^{k + 1}.
        \end{align*}
        \item We only outline the proof. For \(m > 0\), \(n > 0\) this result
        follows directly from the generalized associative law. For \(m < 0\),
        \(n < 0\), we can replace \(a, m, n\) by \(a^{-1}, -m, -n\) respectively
        and argue similarly. The case \(m = 0\) and \(n = 0\) is trivial. The
        case \(m \geq 0\), \(n < 0\) and \(m < 0\), \(n \geq 0\) can be shown by
        induction on \(m\) and \(n\), respectively.
        
        \item TODO

    \end{enumerate}
\end{proof}


\begin{definition}
    A group \(G\) is said to be \emph{abelian} (or \emph{commutative}) if for
    all \(a, b \in G\),
    \[
        ab = ba.
    \]
\end{definition}

\begin{remark}
    As a matter of convention, we shall write
    \[
        a + b \quad \text{instead of} \quad ab
    \]
    in an abelian group and refer to the operation and the term \(a + b\) as
    `addition' and the `sum' of \(a\) and \(b\), respectively. We shall also
    write \(-a\) instead of \(a^{-1}\), \(na\) instead of \(a^n\), and \(0\) for
    the identity element of the group. The sum \(a + (-b)\) for elements \(a,
    b\) of an abelian group will also be abbreviated to \(a - b\). Note that
    \(na\) does not refer to the `group product' in the sense of our definition
    of a group, but is purely a matter of notation.
\end{remark}

\begin{theorem}
    In an abelian group \(G\), \((ab)^n = a^n b^n\) and \((ab)^{-1} = a^{-1}
    b^{-1}\) for all \(a, b \in G\) and all integers \(n\).
\end{theorem}

\begin{definition}
    An element \(a\) of a group \(G\) is said to have \emph{finite order} if
    there exists a positive integer \(n\) such that \(a^n = e\). In this case,
    the smallest such positive integer is called the \emph{order} of \(a\) and
    is denoted by \(\order{a}\). If no such positive integer exists, then \(a\)
    is said to have \emph{infinite order} and we write \(\order{a} = \infty\).
\end{definition}

\begin{theorem}
    If \(g^n = e\) for some integer \(n\), then \(\order{g}\) divides \(n\).
\end{theorem}

\begin{proof}
    From the definition of order, we immediately have \(n \geq \order{g}\). Let
    \(\nu = \order{g}\). By 
\end{proof}


\section{Examples of groups}

\begin{definition}
    Let \(X\) be a set. The set of all bijections from \(X\) to itself forms a
    group under composition of functions, called the \emph{symmetric group} (or
    \emph{permutation group}) on \(X\), denoted by \(S_X\). If \(X = \{1, 2,
    \ldots, n\}\), then we write \(S_n\) instead of \(S_X\).
\end{definition}

\begin{example}[\(\Z\) and \(\Z/n\Z\)]
    \label{def:z-and-zmodn}
    The set of integers \(\Z\) forms a group under the usual addition of
    integers with identity element \(0\). Let \(n\) be a positive integer.
    Recall that congruence modulo \(n\), defined by
    \[
        a \equiv b \pmod{n} \quad \text{if and only if} \quad n \mid (a - b),
    \]
    for all integers \(a, b\), is an equivalence relation on \(\Z\). We define
    \(\Z/n\Z\) to be the set of equivalence classes of this relation. We can
    verify that \(\Z/n\Z\) consists of exactly \(n\) elements, namely
    \[
        [0]_n, [1]_n, \ldots, [n - 1]_n,
    \]
    where \([a]_n\) denotes the equivalence class of \(a\) modulo \(n\). We
    define the group operation on \(\Z/n\Z\) by
    \[
        [a]_n + [b]_n = [a + b]_n.
    \]
    which we call `addition modulo \(n\)'.  First we need to show that this
    operation is well-defined, i.e., the result of the operation does not depend
    on the choice of representatives of the equivalence classes. Suppose that
    \(a \equiv a' \pmod{n}\) and \(b \equiv b' \pmod{n}\). Then \(a = a' + kn\)
    and \(b = b' + \ell n\) for some integers \(k, \ell\). We then have
    \[
        (a' + b') - (a + b) = (a' - a) + (b' - b) = kn + \ell n = (k + \ell)n,
    \]
    and thus
    \[
        (a + b) \equiv (a' + b') \pmod{n}.
    \]
    Associativity follows from the associativity of addition in \(\Z\) and the
    identity element is \([0]_n\), viz.,
    \[
        [a]_n + [0]_n = [a + 0]_n = [a]_n.
    \]
    Similarly for each \([a]_n\) there exists an inverse \([-a]_n\) (which is
    taken to be the equivalence class of \(-a\) modulo \(n\) where \(-a\) is the
    inverse of \(a\) in \(\Z\)) such that
    \[
        [a]_n + [-a]_n = [a - a]_n = [0]_n.
    \]
\end{example}

\begin{example}[Cyclic groups]
    A group \(C\) is said to be \emph{cyclic} if there exists an element \(g \in
    C\) such that
    \[
        C = \{g^k \mid k \in \Z\}.
    \]
    That is, every element of \(C\) is a power of \(g\). In this case, we say
    that \(C\) is \emph{generated} by \(g\), and we write \(C = \gen{g}\). If
    \(g\) has order \(n\) in \(C = \gen{g}\), then \(g^{n+1} = g\), and \(C\)
    must have a finite number of elements, namely
    \[
        C = \{e, g, g^2, \ldots, g^{n - 1}\}.
    \]
    Otherwise, if \(g\) has infinite order in \(C\), then \(C\) must likewise
    have an infinite number of elements. In the first case we call \(C\) the
    \emph{(finite) cyclic group of order \(n\)} and write \(C_n\) instead of
    just \(C\).

    Taking \(g = 1\) and using the additive notation we can write the elements
    of \(C_n\) as
    \[
        C_n = \{0, 1, 2, \ldots, n - 1\} = \Z/n\Z.
    \]
    so that \(\Z/n\Z\) is a cyclic group of order \(n\). In
    \S~\ref{sec:homomorphisms-grp} we shall establish a stronger result that
    relates every cyclic group to \(\Z\) and \(\Z/n\Z\).
\end{example}

\section{Homomorphisms and the category {\normalfont\sffamily Grp}}
\label{sec:homomorphisms-grp}

Recall that we have defined a group to be the pair \((G, \mult_G)\) where \(G\)
is a set and \(\mult_G\) is a binary operation on \(G\). That is, \(\mult_G\) is
a map from \(G \times G\) to \(G\). For two groups \((G, \mult_G)\) and \((H,
\mult_H)\), we would like to define a map
\[
    \phi: (G, \mult_G) \to (H, \mult_H)
\]
that is a function (which with some abuse of notation we shall also denote by
\(\phi\)) between the underlying sets \(G\) and \(H\) that `preserves' the group
structure.

If we define the function
\[
    \phi \times \phi: G \times G \to H \times H
\]
by \((a, b) \mapsto (\phi(a), \phi(b))\), then we get the following diagram:
\[
    \begin{tikzcd}
        G \times G \arrow{r}{\phi \times \phi} \arrow[swap]{d}{\mult_G} & H \times H \arrow{d}{\mult_H} \\
        G \arrow{r}{\phi} & H
    \end{tikzcd}
\]
By requiring that the diagram commute, we obtain a function \(\phi: G \to H\)
that does exactly what we want. Indeed if the diagram commutes, then for all
\(a, b \in G\),
\[
    \begin{tikzcd}
        (a, b) \arrow[swap, mapsto]{d}{\mult_G} & & (a, b) \arrow{r}{\phi \times \phi} & (\phi(a), \phi(b)) \arrow[mapsto]{d}{\mult_H} \\
        a\cdot b \arrow[mapsto]{r}{\phi} & \phi(a\cdot b) && \phi(a)\cdot\phi(b)
    \end{tikzcd}
\]
where \(\cdot\) on the left-hand side is the group operation in \(G\) and
\(\cdot\) on the right-hand side is the group operation in \(H\). Since the
diagram commutes, we must arrive at the same element in \(H\) by following the
two paths. This gives us the following more familiar definition.


\begin{definition}
    Let \(G\) and \(H\) be groups. A \emph{homomorphism} from \(G\) to \(H\) is
    a map \(\phi: G \to H\) that satisfies
    \[
        \phi(ab) = \phi(a)\phi(b)
    \]
    for all \(a, b \in G\).
\end{definition}

We are again being imprecise but we shall not distinguish between the the
multiplications in \(G\) and \(H\) and simply write \(\phi(ab) =
\phi(a)\phi(b)\) when the context is clear.

\begin{definition}
    If \(G\) and \(H\) are groups, we denote the set of group homomorphisms from
    \(G\) to \(H\) as
    \[
        \Hom_{\Grp}(G, H).
    \]
    This allows us to define the category \(\Grp\) whose objects are groups and
    whose morphisms are group homomorphisms as in the following theorem.
\end{definition}

\begin{theorem}
    The category \(\Grp\) is a category whose objects are groups and whose
    morphisms are group homomorphisms.
\end{theorem}

\begin{proof}
    
\end{proof}

It should be noted that we have defined morphisms in the category \(\Grp\) using
only the underlying set and the corresponding group operation. However, our
definition of a group also posits the existence of an identity element and of
inverses. We are able to do so because, as the following theorem shows,
preserving the group operation is enough to preserve the identity element and
inverses.

\begin{theorem}
    A homomorphism preserves the identity element and inverses, i.e., given a
    homomorphism \(\phi: G \to H\),
    \begin{enumerate}[label=(\alph*)]
        \item \(\phi(e_G) = e_H\), and
        \item \(\phi(a^{-1}) = \phi(a)^{-1}\),
    \end{enumerate}
    where \(e_G\) and \(e_H\) are the identity elements of \(G\) and \(H\)
    respectively, and \(a \in G\).
\end{theorem}

\begin{proof}
    We prove each part in turn.

    \begin{enumerate}[label=(\alph*), wide]
        \item We have \(\phi(e_G) = \phi(e_G e_G) = \phi(e_G) \phi(e_G)\). This
        implies that \(\phi(e_G) = e_H\).
        \item Multiplying \(\phi(a)\) by \(\phi(a^{-1})\) gives us
        \(\phi(a)\phi(a^{-1}) = \phi(aa^{-1}) = \phi(e_G) = e_H\). This implies
        that \(\phi(a^{-1}) = \phi(a)^{-1}\).
    \end{enumerate}
\end{proof}

The second assertion in the preceding theorem is equivalent to saying that the
diagram
\[
    \begin{tikzcd}
        G \arrow{r}{\phi} \arrow[swap]{d}{\id_G} & H \arrow{d}{\id_H} \\
        G \arrow[swap]{r}{\phi} & H
    \end{tikzcd}
\]
commutes. (Here \(\id_G\) and \(\id_H\) are the identity maps on \(G\) and \(H\)
respectively.)

Having shown that \(\Grp\) is a category, we now ask whether products and
coproducts exist in \(\Grp\). If \(G\) and \(H\) are groups, a natural place to
start is the product \(G \times H\) of their underlying sets in the category
\(\catg{Set}\). We can endow this set with a group structure by defining the
group operation componentwise, i.e., for \(g_1, g_2 \in G\) and \(h_1, h_2 \in
H\),
\[
    (g_1, h_1) \cdot (g_2, h_2) = (g_1g_2, h_1h_2).
\]
This operation is associative because the group operations in \(G\) and \(H\)
are associative. The identity element is \((e_G, e_H)\) and the inverse of \((g,
h)\) is \((g^{-1}, h^{-1})\). The set \(G \times H\) with this group operation
is called the \emph{direct product} of \(G\) and \(H\). From this definition it
also follows that the natural projections
\[
        \begin{tikzcd}
            & G\times H \arrow[ld, "\pi_G"'] \arrow[rd, "\pi_H"] &   \\
          G &                                                    & H
          \end{tikzcd}
\]
defined as set functions are group homomorphisms. We can now establish the
following result.

\begin{theorem}
    The direct product \(G \times H\) of two groups \(G\) and \(H\) is a product
    in the category \(\Grp\).
\end{theorem}

\begin{proof}
    We need to show that \(G \times H\) satisfies the universal property of a
    product: i.e., for any group \(K\) and group homomorphisms \(\phi_G: K \to
    G\) and \(\phi_H: K \to H\), there exists a unique group homomorphism
    \(\Phi: K \to G \times H\) such that the following diagram commutes:
    \[
        \begin{tikzcd}
            &  &                                                    & G \\
        K \arrow[rr, "\Phi"] \arrow[rrru, "\phi_G", bend left] \arrow[rrrd, "\phi_H"', bend right] &  & G \times H \arrow[rd, "\pi_H"] \arrow[ru, "\pi_G"] &   \\
                    &  &                                                    & H
        \end{tikzcd}
    \]
    Since \(G \times H\) is a product in \(\catg{Set}\), there exists a set
    function \(\Phi: K \to G \times H\) such that \(\pi_G \circ \Phi = \phi_G\)
    and \(\pi_H \circ \Phi = \phi_H\). This function \(\Phi\) is defined as
    \[
        \Phi(k) = (\phi_G(k), \phi_H(k)), \quad \text{for all} \quad k \in K.
    \]
    It would suffice to show that \(\Phi\) is a group homomorphism. We have for
    all \(k_1, k_2 \in K\),
    \begin{align*}
        \Phi(k_1k_2) & = (\phi_G(k_1k_2), \phi_H(k_1k_2))\\
        & = (\phi_G(k_1)\phi_G(k_2), \phi_H(k_1)\phi_H(k_2))\\
        & = (\phi_G(k_1), \phi_H(k_1))(\phi_G(k_2), \phi_H(k_2))\\
        & = \Phi(k_1)\Phi(k_2).
    \end{align*}
    Therefore \(\Phi\) is a group homomorphism. The uniqueness of \(\Phi\) is a
    consequence of the uniqueness of the set function \(\Phi\) in the category
    \(\catg{Set}\). Therefore \(G \times H\) is a product in \(\Grp\).
\end{proof}


\begin{example}
    We return to \(\Hom(G, H)\). Clearly \(\Hom(G, H)\) is non-empty because we
    can construct at the very least a homomorphism that maps every element of
    \(G\) to the identity element of \(H\). Alternatively, from a
    category-theoretic perspective, the fact thet the trivial group are initial
    and terminal objects in \(\Grp\) allows us to define the unique morphisms
    \[
        G \to T \quad \text{and} \quad T \to G
    \]
    (where \(T\) is the trivial group) whose composition is exactly the
    homomorphism that maps every element of \(G\) to the identity element of
    \(H\).
\end{example}

\begin{example}
    Consider the dihedral group \(D_6\) (recall that \(D_6\) is the group of
    symmetries of a regular triangle) and the map \(D_6 \to S_3\) that sends
    each symmetry of the triangle to the corresponding permutation of the
    vertices. We claim that this map is a group homomorphism.
\end{example}

\begin{example}
    The exponential function is a homomorphism from the group of real numbers
    under addition to the group of positive real numbers under multiplication.
    Indeed,
    \[
        \exp (a + b) = \exp a \exp b.
    \]
    A similar class of examples can be given by defining the map \(\epsilon_g :
    \Z \to G\) for some group \(G\) by \(\epsilon_g(n) = g^n\) for some \(g \in
    G\). This is a homomorphism because
    \[
        \epsilon_g(m + n) = g^{m + n} = g^m g^n = \epsilon_g(m) \epsilon_g(n).
    \]
    The image of \(\epsilon_g\) is a cyclic subgroup of \(G\) generated by
    \(g\). Therefore if \(\epsilon_g\) is surjective, then \(G\) itself must be
    cyclic and generated by \(g\).
\end{example}

\begin{example}
    Let \(n\) be a positive integer. The map \(\Z \to \Z/n\Z\) defined by \(a
    \mapsto [a]_n\) for all \(a \in G\) (and where \([a]_n\) is the equivalence
    class of \(a\) modulo \(n\)) is a group homomorphism. Since this is
    equivalent to sending \(a\) to \(a \cdot [1]_n\) we see that this is the
    same as the map \(\epsilon_{[1]_n}\) in the preceding example.
\end{example}

\begin{remark}
    \label{rem:homomorphism-order}
    As our motivation for defining a group homomorphism was the desire to find a
    mapping that preserves the group structure, it should be natural to ask
    whether it would also preserve the order of an element in a group. If we
    consider an element \(g\) of order \(n\) in a group \(G\) and a group
    homomorphism \(\phi: G \to H\), we can establish by induction that
    \[
        \phi(g)^n = \phi(g^n) = \phi(e_G) = e_H.
    \]
    More precisely, we have:
\end{remark}

\begin{theorem}
    \label{thm:order-divides-n-homomorphism}
    Let \(\phi: G \to H\) be a group homomorphism and let \(g \in G\) be an
    element of finite order. Then \(\order{\phi(g)}\) divides \(\order{g}\).
\end{theorem}

\begin{proof}
    Since \(\phi(g)^{\order{g}} = e_H\) as we have remarked in
    \S~\ref{rem:homomorphism-order}, Theorem~\ref{thm:order-divides-n} gives us
    this result.
\end{proof}

\begin{example}
    There are no nontrivial homomorphisms from the group of integers under
    addition modulo \(n\) to the additive group of integers. Since every element
    of \(\Z/n\Z\) has finite order, the preceding theorem implies that the image
    of any homomorphism from \(\Z/n\Z\) to \(\Z\) must be the identity \(0\) as
    it is the only element of finite order in \(\Z\).
\end{example}

\begin{example}
    It is important to note that Theorem~\ref{thm:order-divides-n-homomorphism}
    does not imply that the order itself is preserved. As a counterexample,
    consider the element \(1 \in \Z\) and the map \(\pi_n : \Z \to \Z/n\Z\)
    defined by \(a \mapsto [a]_n\). The order of \(1\) in \(\Z\) is infinite,
    but the order of \(\pi_n(1)\) in \(\Z/n\Z\) is \(n\). Isomorphisms, however,
    do preserve the order of elements, as we shall see in the following
    sections.
\end{example}

\begin{theorem}
    An isomorphism of groups \(\phi: G \to H\) is an isomorphism in the category
    \(\Grp\), i.e., it admits an inverse
    \[
        \phi^{-1}: H \to G
    \]
    that is also a group homomorphism.
\end{theorem}

\begin{proof}
    Suppose that \(\phi: G \to H\) is an isomorphism of groups (i.e., it is
    bijective as a map between sets). Then it has an inverse \(\phi^{-1}: H \to
    G\) that is also a map between sets. We claim that \(\phi^{-1}\) is a group
    homomorphism. Let \(a, b \in H\). Then there exist \(x, y \in G\) such that
    \(\phi(x) = a\) and \(\phi(y) = b\). We have
    \[
        \phi^{-1}(ab) = \phi^{-1}(\phi(x)\phi(y)) = \phi^{-1}(\phi(xy)) = xy = \phi^{-1}(a)\phi^{-1}(b).
    \]
    
    Conversely, suppose that \(\phi\) is a group homomorphism that admits an
    inverse \(\phi^{-1}\) that is also a group homomorphism. Then \(\phi\) is
    bijective as a map between sets and thus an isomorphism of groups.
\end{proof}


\begin{theorem}
    Let \(\phi: G \to H\) be an isomorphism. Then \(\order{\phi(g)} =
    \order{g}\) for all \(g \in G\).
\end{theorem}

\begin{theorem}
    Let \(\phi : G \to H\) be an isomorphism. Then \(G\) is abelian if and only
    if \(H\) is abelian.
\end{theorem}

\begin{proof}
    Suppose that \(G\) is abelian. Then for all \(a, b \in H\), we have
    \[
        \phi(a)\phi(b) = \phi(ab) = \phi(ba) = \phi(b)\phi(a),
    \]
    so \(H\) is abelian. The proof for the converse is similar.
\end{proof}

\section{Free groups}

Given some set \(X\) that does not necessarily have any group structure, we
would now like to construct another group \(F\) containing \(X\) as a subset in
the `most efficient' way possible

\begin{definition}
    The group \(F\) is a \emph{free group} on the set \(X\) (or alternatively,
    we say that \(F\) is \emph{free} on \(X\)) if there is a set function
    \(\iota: X \to F\) such that for any group \(G\) and any set function
    \(\phi: X \to G\), there exists a unique group homomorphism \(\Phi: F \to
    G\) such that \(\Phi \circ \iota = \phi\), i.e., the following diagram
    commutes:
    \[
        \begin{tikzcd}
            X \arrow[d, "\iota"'] \arrow[r, "\phi"] & G \\
            F \arrow[ru, "\exists!\Phi"', dashed]        &  
        \end{tikzcd}
    \]
\end{definition}

This means that every function from a set \(X\) to a group \(G\) can be
decomposed into a function from \(X\) to \(F\) followed by a homomorphism from
\(F\) to \(G\). (An alternative notation is to write \(F(X)\) to emphasize that
\(F\) is the free group on \(X\).)

If such a group exists, it is unique up to isomorphism. Consider for example a
set \(X\) and a set \(G\) and suppose that \(F_1\) and \(F_2\) are groups
satisfying the universal property in our definition. Then because \(F_1\) is
free on \(X\) and \(F_2\) is a group (and vice versa), we have:
\[
    \begin{tikzcd}
        & X \arrow[dl, "\iota_1"'] \arrow[dr, "\iota_2"] & \\
        F_1 \arrow[rr, "\exists!\Phi", shift left] & & F_2 \arrow[ll, "\exists!\Psi", shift left]
    \end{tikzcd}
\]
where \(\Phi\) and \(\Psi\) are group homomorphisms whose existence is
guaranteed by the universal property for \(F_1\) and \(F_2\), respectively. We
then have
\[
    \Phi \circ \iota_1 = \iota_2 \quad \text{and} \quad \Psi \circ \iota_2 = \iota_1.
\]
Substituting \(\iota_1\) in the first equation with \(\Psi \circ \iota_2\) (and
vice versa) gives us
\[
    \Phi \circ \Psi \circ \iota_2 = \iota_2 \quad \text{i.e.,} \quad \Phi \circ \Psi = \id_{F_2}
\]
and
\[
    \Psi \circ \Phi \circ \iota_1 = \iota_1 \quad \text{i.e.,} \quad \Psi \circ \Phi = \id_{F_1}.
\]
That is \(\Phi\) and \(\Psi\) are inverses of each other, so \(F_1\) and \(F_2\)
are isomorphic.

An alternative characterization would be to consider for a set \(X\) the
category \(\catg{F}^X\) whose objects are pairs \((k, G)\) where each \(k\) is a
set function \(X \to G\) and each \(G\) is a group, and whose morphisms are
commutative diagrams of set functions of the form
\[
    \begin{tikzcd}
        G_1 \arrow[r, "\phi"]                 & G_2 \\
        X \arrow[u, "k_1"] \arrow[ru, "k_2"'] &    
    \end{tikzcd}
\]
where again each \(k_i\) is a set function \(X \to G_i\) and each \(G_i\) is a
group, and where \(\phi\) is a group homomorphism. Since we are considering all
set functions \(X \to G\), we are not encoding any group structure in the
objects of \(\catg{F}^X\). The free group \(F\) is then the group component of
an initial object (which we write \((k_0, F)\)) in \(\catg{F}^X\).

Because \((k_0, F)\) is initial in \(\catg{F}^X\), there is a unique morphism
from \((k_0, F)\) to any other object \((k, G)\) in \(\catg{F}^X\). This
morphism is a group homomorphism \(\Phi: F \to G\) such that \(\Phi \circ k_0 =
k\). This is the same as saying that \(F\) is free on \(X\) (based on our
earlier definition). Moreover, because \((k_0, F)\) is initial, it is unique up
to isomorphism.

An initial object in a category need not necessarily exist. We shall now show
however that there exists an initial object in the category \(\catg{F}^X\). 

Consider a set \(X\) and let \(X^{-1}\) be an isomorphic copy of \(X\) with \(X
\cap X^{-1} = \emptyset\). Since \(X \cong X^{-1}\), we can choose a bijection
\(\phi : X \to X^{-1}\) and write \(x^{-1}\) for \(\phi(x)\) for all \(x \in
X\). (If \(x^{-1} X^{-1}\) then the preimage of \(x^{-1}\) is \(x\) and thus
\(x^{**} = x\) by definition.) We define a \emph{word} in \(X\) to be a finite
\(n\)-tuple of elements of \(x_i \in X \cup X^{-1}\) of the form
\[
    (x_1, x_2, \ldots, x_n)
\]
which we will denote by juxtaposition
\[
    w = x_1 x_2 \cdots x_n.
\]
We shall call \(X \cup X^{-1}\) the \emph{alphabet} on \(X\) and each \(a_i\) a
\emph{letter} on this alphabet. The number \(n\) is the \emph{length} of the
word \(w\). For example if \(X = \{a, b\}\), then
\[
    abba^{-1}b^{-1}a^{-1}
\]
is a word of length \(6\) on the alphabet \(\{a, b, a^{-1}, b^{-1}\}\). For each
\(a\) we call \(a^{-1}\) the \emph{inverse} of \(a\), noting however that our
usage here does not necessarily coincide with the group-theoretic notion of an
inverse. (We shall later show that for each letter \(x\), its inverse is exactly
\(x^{-1}\) but we shall stop short of this claim for now.) Write \(W(X)\) for
the set of all words in \(X\), with the empty word (written \(\epsilon\))
defined as the set containing no letters.

The concatenation of two words \(w_1 = x_1 x_2 \cdots x_m\) and \(w_2 = y_1 y_2
\cdots y_n\) is the word
\[
    w_1 w_2 = x_1 x_2 \cdots x_m y_1 y_2 \cdots y_n.
\]
A \emph{substring} of a word \(w = x_1 x_2 \cdots x_n\) is a word of the form
\(x_i x_{i + 1} \cdots x_j\) for some \(i, j\) with \(1 \leq i \leq j \leq n\).
We can immediately observe that the empty word is a substring of any word. We
shall call a word \(w = x_1 x_2 \cdots x_n\) \emph{reduced} if it contains no
substrings of the form \(aa^{-1}\) or \(a^{-1}a\). For example, the word
\(abb^{-1}a\) is not reduced, but the word \(ab\) is.


Now we want to introduce a process through which, say two words \(bab^{-1}bb\)
and \(babb^{-1}baa^{-1}\) in \(X = \{a, b\}\) both simplify to the same word
\(bab\), i.e., we want the `inverse' \(a^{-1}\) of \(a\) to behave in a similar
manner as a group inverse (in the limited sense that inverses `cancel out' each
other). To do this, we first define a map \(W(X) \to W(X)\) that deletes from a
given word \(w\) any pair of the form \(aa^{-1}\) or \(a^{-1}a\). We call the
image of \(w\) under this map an \emph{elementary reduction} of \(w\). Our goal
then is to find for each word \(w\) in \(X\) a sequence of elementary reductions
\[
    w = w_0 \mapsto w_1 \mapsto w_2 \mapsto \cdots \mapsto w_n
\]
where \(w_n\) is a word in \(X\) that cannot be further reduced, i.e., any
elementary reduction of \(w_n\) is equal to \(w_n\). (That is, \(w_n\) is a
reduced word by our earlier definition.) For each word \(w\), at least one such
sequence of elementary reductions exists, which we obtain by considering only
reductions that delete the first occurrence of a pair of the form \(aa^{-1}\) or
\(a^{-1}a\) from left to right. This must terminate after \(\lfloor \frac{n}{2}
\rfloor\) steps, where \(n\) is the length of \(w\) as each elementary reduction
of this type decreases the length of the word by \(2\). For the more general
case we refer to the following result, often called the `diamond lemma.'

\begin{lemma}
    If \(w \mapsto w_1\) and \(w \mapsto w_2\) are two elementary reductions of
    a word \(w\) in \(X\), then there exists elementary reductions \(w_1 \mapsto
    w_0\) and \(w_2 \mapsto w_0\) of \(w_1\) and \(w_2\) such that the following
    diagram commutes:
    \[
        \begin{tikzcd}
            & w \arrow[ld, maps to] \arrow[rd, maps to] &                \\
            w_1 \arrow[rd, maps to] &                         & w_2 \arrow[ld, maps to] \\
                        & w_0                     &               
        \end{tikzcd}
    \]
\end{lemma}

\begin{proof}
    Let \(\rho_1 : w \mapsto w_1\) and \(\rho_2 : w \mapsto w_2\) be elementary
    reductions of \(w\). Let \(w = u\omega v\) be a word in \(X\) where \(u, v\)
    are (possibly empty) substrings of \(w\) and \(\omega \in X \cup X^{-1}\) is
    the substring undergoing an elementary reduction, i.e., \(\omega\) must be
    of the form \(aa^{-1}\) or \(a^{-1}a\) or the empty string and thus the
    reduction is given by \(u\omega v \mapsto uv\).

    We consider two possible cases. If \(\rho_1\) and \(\rho_2\) are disjoint,
    i.e., if they affect disjoint substrings \(\omega_i\) of \(w\) (affected by
    the \(\rho_i\)), then \(w\) must be of the form \(u\omega_1v\omega_2y\)
    where \(u, v, y\) are (possibly empty) substrings of \(w\), and composing
    the elementary reductions gives us
    \[
        \rho_1 \circ \rho_2 : w \mapsto u\omega_1vy \mapsto uvy
    \]
    and 
    \[
        \rho_2 \circ \rho_1 : w \mapsto uv\omega_2y \mapsto uvy,
    \]
    and the lemma holds.

    On the other hand, if \(\rho_1\) and \(\rho_2\) are not disjoint, then \(w\)
    must be of the form \(u\omega v\) where \(u, v\) are (possibly empty)
    substrings of \(w\) and \(\omega\) is the substring affected by both
    \(\rho_1\) and \(\rho_2\) which must then be of the form \(a^{-1}aa^{-1}\)
    or \(aa^{-1}a\). Without loss of generality, suppose that \(\omega =
    a^{-1}aa^{-1}\) and \(\rho_1\) reduces \(a^{-1}a\) and \(\rho_2\) reduces
    \(aa^{-1}\). Then we have
    \[
        \rho_1 : w = u(a^{-1}a)a^{-1}v \mapsto ua^{-1}v
    \]
    and
    \[
        \rho_2 : w = ua^{-1}(aa^{-1})v \mapsto ua^{-1}v
    \]
    and thus the lemma also holds.
\end{proof}

\begin{theorem}
    Let \(w\) be a word in a set \(X\). Then any sequence of elementary
    reductions
    \[
        w \mapsto w_1' \mapsto w_2' \mapsto \cdots \mapsto w_n'
    \]
    and
    \[
        w  \mapsto w_1'' \mapsto w_2'' \mapsto \cdots \mapsto w_m''
    \]
    must terminate with the same reduced word \(w_n = w_m'\).
\end{theorem}

\begin{proof}
    We prove this by induction on the length \(k\) of the word \(w\). If \(k =
    0\), then the word is already reduced and there is nothing to prove.
    Similarly if \(k = 1\) then \(w\) is a letter and is already reduced. Now by
    the diamond lemma, there exists elementary reductions \(w'_1 \mapsto w_1\)
    and \(w''_1 \mapsto w_1\). Consider the sequence of elementary reductions
    \[
        w_1 \mapsto w_2 \mapsto \cdots \mapsto w_j
    \]
    where each \(w_i\) is obtained by applying the lemma. That is, since \(w_1\)
    is a reduction of both \(w_1'\) and \(w_1''\), it must follow that \(w_2'\)
    and \(w_2''\) (being reductions of \(w_1'\) and \(w_1''\)) must reduce to
    the same word \(w_2\). By induction, we therefore have \(w'_n = w_j =
    w_m''\).
\end{proof}

Write \(\overline{w}\) for the reduced word obtained from a word \(w\) by a
sequence of elementary reductions (this must exist and be unique by the results
we have established thus far). Let \(F\) be the set of all reduced words in
\(X\). Then the empty word \(\epsilon\) is a reduced word and must be in \(F\).
Define a binary operation \(\cdot\) on \(F\) by
\[
    w_1 \cdot w_2 = \overline{w_1 w_2}
\]
where \(w_1, w_2 \in F\) and \(w_1 w_2\) is the concatenation of \(w_1\) and
\(w_2\). We claim that \((F, \cdot)\) is a group under this operation.

The empty word \(\epsilon\) is the identity element of \(F\) under this
operation. Indeed, for any word \(w\), we have
\[
    w \cdot \epsilon = \overline{w \epsilon} = \overline{w} = w = \overline{\epsilon w} = \epsilon \cdot w.
\]
We define the inverse of a word \(w\) to be the word \(w^{-1}\) obtained by
reversing the order of the letters in \(w\) and replacing each letter by its
inverse. That is, if \(w = x_1 x_2 \cdots x_n\) is a word, then its inverse is
given by
\[
    w^{-1} = x_n^{-1} x_{n - 1}^{-1} \cdots x_1^{-1},
\]
and thus
\[
    w \cdot w^{-1} = x_1 \cdots x_n x_n^{-1} \cdots x_1 = \epsilon.
\]
(We can show the same for \(w^{-1}w\).) Finally, we want to show that the
operation \(\cdot\) is associative. Let \(w_1, w_2, w_3\) be words in \(F\).
Then since \(\overline{(\overline{w_1 w_2})w_3}\) and
\(\overline{w_1(\overline{w_2 w_3})}\) can be obtained from \(w_1 w_2 w_3\) by a
sequence of elementary reductions, we have
\[
    \overline{\overline{w_1 w_2} w_3} = \overline{w_1 w_2 w_3} = \overline{w_1 \overline{w_2 w_3}},
\]
and thus the operation \(\cdot\) is associative. This completes the proof that
\((F, \cdot)\) is a group.

\begin{theorem}
    Let \(X\) be a set and let \(F\) be the set of all reduced words in \(X\)
    with the operation \(\cdot\) defined as above. Then \(F\) is the free group
    on \(X\). 
\end{theorem}

\begin{proof}
    We need to show that \(F\) satisfies the universal property for free groups
    on the set \(X\), i.e., there exists a map \(\iota: X \to F\) such that for
    any group \(G\) and any set function \(\phi: X \to G\), there exists a
    unique group homomorphism \(\Phi: F \to G\) such that \(\Phi \circ \iota =
    \phi\).

    The map \(\iota: X \to F\) is the inclusion map that sends each element of
    \(X\) to the corresponding letter in \(F\). (From this we observe that \(X\)
    is contained in \(F\) as we intended it to be at the beginning of this
    section.) Let \(G\) be a group and let \(\phi: X \to G\) be a set function;
    we want to construct a group homomorphism \(\Phi: F \to G\) such that the
    diagram
    \[
        \begin{tikzcd}
            X \arrow[d, "\iota"'] \arrow[r, "\phi"] & G \\
            F \arrow[ru, "\exists!\Phi"', dashed]        &  
        \end{tikzcd}
    \]
    commutes. Since \(F\) is the set of reduced words in \(X\), each \(w \in F\)
    is of the form
    \[
        w = x_1^{\epsilon_1} x_2^{\epsilon_2} \cdots x_n^{\epsilon_n}
    \]
    where each \(x_i \in X\) and each \(\epsilon_i = \pm 1\) (in this notation,
    \(x^{-1} = x^{-1}\)). We define \(\Phi(w)\) to be the product
    \[
        \Phi(w) = \phi(x_1)^{\epsilon_1} \phi(x_2)^{\epsilon_2} \cdots \phi(x_n)^{\epsilon_n}.
    \]
    The expression \(\phi(x)^{-1}\) is defined because \(\phi(x) \in G\) and
    inverses exist in \(G\).
    
    We claim that \(\Phi\) is a group homomorphism. We identify with \(X\) its
    image under \(\iota\), i.e., we take \(X\) to be a subset of the underlying
    set of \(F\). To prove our claim, we first need to show that if \(w\) is a
    reduced word and \(x \in X\), then \(\Phi(x\cdot w) = \phi(x)\Phi(w)\) and
    \(\Phi(x^{-1} \cdot w) = \phi(x)^{-1}\Phi(w)\). From this it would then
    follow, using the fact that the operation \(\cdot\) on \(F\) is associative,
    and by induction on the length of the reduced word \(w\), that for all
    reduced words \(w_1\) and \(w_2\), we have \(\Phi(w_1 \cdot w_2) = \Phi(w_1)
    \Phi(w_2)\) and thus \(\Phi\) is a group homomorphism.

    Write the reduced word \(w\) as \(w = x_1^{\epsilon_1} x_2^{\epsilon_2}
    \cdots x_r ^{\epsilon_r}\) where each \(x_i \in X\) and each \(\epsilon_i =
    \pm 1\) for some \(r \in \N\). If \(x \neq x_1\), or if \(x_1 = x\) and
    \(\epsilon_1 = 1\), then \(x \cdot w\) must be reduced and we have
    \begin{align*}
        \Phi(x)\Phi(w) & \phi(x)(\phi(x_1)^{\epsilon_1} \cdots \phi(x_r)^{\epsilon_r}) = \phi(x)\phi(x_1)^{\epsilon_1} \cdots \phi(x_r)^{\epsilon_r} \\
                       & = \phi(x x_1^{\epsilon_1} \cdots x_r^{\epsilon_r}) = \Phi(x \cdot w).
    \end{align*}
    If \(x = x_1\) and \(\epsilon_1 = -1\), then  \(x \cdot w\) is the reduced
    word \(x_2^{\epsilon_2} \cdots x_r^{\epsilon_r}\), so that
    \begin{align*}
        \Phi(x \cdot w) & = \Phi(x_2^{\epsilon_2} \cdots x_r^{\epsilon_r}) = \phi(x_2)^{\epsilon_2} \cdots \phi(x_r)^{\epsilon_r} \\
            & = \phi(x_1)\phi(x_1)^{-1}\phi(x_2)^{\epsilon_2} \cdots \phi(x_r)^{\epsilon_r} \\ &= \phi(x)(\phi(x_1)^{-1}\phi(x_2)^{\epsilon_2} \cdots \phi(x_r)^{\epsilon_r}) \\
            & = \Phi(x)\Phi(w),
    \end{align*}
    as desired. A similar argument establishes \(\Phi(x^{-1} \cdot w) =
    \Phi(x)^{-1}\Phi(w)\).

    Now let \(w_1\), \(w_2\) be reduced words in \(F\) and let \(w_1\) have
    length \(n\). We proceed by induction on \(n\). The case \(n = 1\) is the
    result we have just established. Suppose that the result holds for \(w_1\)
    of length \(k\) and suppose that \(w_1\) is of the form
    \[
        w_1 = x_0^{\epsilon_0} x_1^{\epsilon_1} \cdots x_k^{\epsilon_k}.
    \]
    Then \(w_1w_2 = x_0^{\epsilon_0} (w'_1 w_2)\) where \(w'_1 =
    x_1^{\epsilon_1} \cdots x_k^{\epsilon_k}\). Since \(w'_1w_2\) need not
    necessarily be reduced, let \(w_3 = \overline{w'_1 w_2}\) so that \(w_1
    \cdot w_2 = x_0^{\epsilon_0} \cdot w_3\). By the inductive hypothesis, we
    have \(\Phi(w_3) = \Phi(w'_1) \Phi(w_2)\) and thus we have
    \begin{align*}
        \Phi(w_1 \cdot w_2) & = \Phi(x_0^{\epsilon_0}(\overline{w'_1 w_2})) = \Phi(x_0^{\epsilon_0} \cdot w_3) = \Phi(x_0^{\epsilon_0}) \Phi(w_3) \\
                            & = \Phi(x_0^{\epsilon_0})\left(\Phi(w'_1) \Phi(w_2)\right) = \left(\Phi(x_0^{\epsilon_0}) \Phi(w'_1)\right) \Phi(w_2) \\
                            & = \Phi(x_0^{\epsilon_0} \cdot w'_1) \Phi(w_2) = \Phi(w_1) \Phi(w_2).
    \end{align*}
    This shows that \(\Phi\) is a group homomorphism. By construction, \(\Phi(x)
    = \phi(x)\) for all \(x \in X\), and \(\Phi\) is uniquely determined by
    \(\phi\). This completes the proof that \(F\) is the free group on \(X\).
\end{proof}

\begin{example}
    Consider the set \(X = \{x\}\). We claim that \(F \cong \Z\). By the
    universal property for free groups on \(X\), there exists a map \(\iota : X
    \to F\) such that for any group \(G\) and any set function \(\phi : X \to
    G\), there exists a unique group homomorphism \(\Phi : F \to G\) such that
    \(\Phi \circ \iota = \phi\). Since \(X\) is a singleton, the map \(\iota\)
    is the inclusion map that sends \(x\) to its image in \(F\), i.e., as a
    letter in the alphabet on \(X\). Let \(G\) be a group and let \(\phi : X \to
    G\) be a set function. Again since \(X\) is a singleton, \(\phi\) is
    determined by the image of \(x\) in \(G\) and \(\phi(x) = g\) for some \(g
    \in G\).

    From the previous theorem, \(F\) is the set of all reduced words in \(X\)
    with the operation \(\cdot\) defined above. The alphabet on \(X\) is \(\{x,
    x^{-1}\}\) and each word in \(W(X)\) must then be of the form  \(x^n\) for
    some \(n \in \Z\). Thus \(F\) is the cyclic group generated by \(x\), i.e.,
    \[
        F = \{x^n : n \in \Z\} = \{\dots, x^{-2}, x^{-1}, x_0, x, x^2, \dots\},
    \]
    where \(x_0 = e\) is the identity element of \(F\), i.e., the empty word. We
    define a map \(\Phi : F \to G\) by \(\Phi(x^n) = g^n\). This is a group
    homomorphism because
    \[
        \Phi(x^m \cdot x^n) = \Phi(x^{m + n}) = g^{m + n} = g^m g^n = \Phi(x^m) \Phi(x^n).
    \]
    Moreover, \(\Phi \circ \iota = \phi\) because \(\Phi(\iota(x)) = \Phi(x) = g
    = \phi(x)\).
\end{example}

\section{Subgroups}

Another way to construct a group is to look at subsets of a group that are
themselves groups. Consider a group \((G, \mult)\) and a subset \(H\) of the
underlying set \(G\). Suppose further that \((H, \mult')\) is itself a group.
How are the group operations \(\mult\) and \(\mult'\) related? We first consider
the following definition.

\begin{definition}
    Let \((G, \mult)\) and \((H, \mult')\) be groups such that \(H \subset G\).
    We say that \(H\) is a \emph{subgroup} of \(G\) if the inclusion function
    \(\iota : H \to G\) is a group homomorphism. If \(H\) is a subgroup of
    \(G\), we write \(H < G\).
\end{definition}

Let us first fix some notation and write \(a \cdot b\) for \(\mult(a,b)\) and
\(a * b\) for \(\mult'(a,b)\). Then because \(\iota\) is a group homomorphism,
we have
\[
    \iota(a * b) = \iota(a) \cdot \iota(b).
\]
Moreover, because \(\iota\) is a group homomorphism, it must preserve the
identity element and inverses. Thus \(\iota(e_H) = e_G\) and \(\iota(a^{-1}) =
\iota(a)^{-1}\) for all \(a \in H\). We can thus see that the group operation
\(\mult'\) in \(H\) is the restriction of the group operation \(\mult\) in \(G\)
to \(H\). Thus, we can alternatively characterize a subgroup as a subset \(H\)
of a group \(G\) that is itself a group under the
group operation of \(G\) (retricted to \(H\)). The next result is a useful
criterion for determining whether a subset is a subgroup.

\begin{theorem}[Subgroup test]
    \label{thm:subgroup-test}
    A nonempty subset \(H\) of a group \(G\) is a subgroup if and only if for
    all \(a, b \in H\), we have \(ab^{-1} \in H\).
\end{theorem}

\begin{proof}
    The forward implication follows directly from the definition. Conversely,
    suppose that \(H\) is a nonempty subset of \(G\) such that \(ab^{-1} \in H\)
    for all \(a, b \in H\). Then \(e = aa^{-1} \in H\) for all \(a \in H\).
    Moreover, if \(a \in H\), then \(a^{-1} = ea^{-1} \in H\). Finally, if \(a,
    b \in H\), then \(ab = a(b^{-1})^{-1} \in H\). Thus \(H\) is a subgroup of
    \(G\).
\end{proof}

\begin{theorem}
    \label{thm:intersection-subgroups}
    If \(\{H_{\alpha}\}_{\alpha \in A}\) is any collection of subgroups of a
    group \(G\), then the intersection
    \[
        H = \bigcap_{\alpha \in A} H_{\alpha}
    \]
    is a subgroup of \(G\). Moreover \(H\) is the largest subgroup of \(G\)
    contained in each of the subgroups \(H_{\alpha}\).
\end{theorem}

\begin{proof}
    The intersection \(H\) is nonempty because \(e \in H_\alpha\) for all
    \(\alpha\). If \(a, b \in H\), then \(a, b \in H_\alpha\) for all \(\alpha
    \in A\). Thus \(ab^{-1} \in H_\alpha\) for all \(\alpha \in A\), so
    \(ab^{-1} \in H\).

    To show that \(H\) is the largest subgroup of \(G\) contained in each of the
    subgroups \(H_{\alpha}\), suppose that \(K\) is a subgroup of \(G\) such
    that \(K \subset H_{\alpha}\) for all \(\alpha \in A\). Then by our
    definition of \(H\), we have \(K \subset H\).
\end{proof}

\begin{theorem}
    Let \(\phi: G \to H\) be a group homomorphism and let \(K\) be a subgroup of
    \(H\). Then \(\phi^{-1}(H)\) is a subgroup of \(G\).
\end{theorem}


Every group \(G\) has at least two subgroups: the trivial subgroup \(\{e\}\) and
\(G\) itself. Consider the homomorphism \(\phi: G \to H\). Define
\[
    \ker \phi = \{g \in G : \phi(g) = e_G\} = \phi^{-1}(\{e_H\}).
\]
and
\[
    \img \phi = \{h \in H : h = \phi(g) \text{ for some } g \in G\}.
\]
Since \(\{e_H\}\) is a subgroup of \(H\), we have by the preceding theorem that
\(\ker \phi\) is a subgroup of \(G\). For a more explicit proof, observe that
\(\ker \phi\) is not empty (because at the very least \(e_G \in \ker \phi\)) and
that if \(a, b \in \ker \phi\), then 
\[
    \phi(ab^{-1}) = \phi(a)\phi(b)^{-1} = e_H e_H^{-1} = e_H,
\]
so \(ab^{-1} \in \ker \phi\) and by the subgroup test, \(\ker \phi\) is a
subgroup of \(G\).

On the other hand, observe that \(\img \phi\) is likewise not empty (because it
contains \(e_H\)). If \(a, b \in \img \phi\), then there exist \(x, y \in G\)
such that \(\phi(x) = a\) and \(\phi(y) = b\). Then
\[
    ab^{-1} = \phi(x)\phi(y)^{-1} = \phi(xy^{-1}),
\]
so \(ab^{-1} \in \img \phi\). Thus \(\img \phi\) is a subgroup of \(H\).


\section{Quotient groups and the isomorphism theorems}

\begin{definition}
    Let \(H\) be a subgroup of a group \(G\) and \(a, b \in G\). We say that
    \(a\) is \emph{left congruent} to \(b\) modulo \(H\) if \(a^{-1}b \in H\)
    and write \(a \equiv_L b \pmod{H}\). Similarly, we say that \(a\) is
    \emph{right congruent} to \(b\) modulo \(H\) if \(ba^{-1} \in H\) and write
    \(a \equiv_R b \pmod{H}\). If \(a \equiv_L b \pmod{H}\) and \(a \equiv_R b
    \pmod{H}\), then we say that \(a\) is \emph{congruent} to \(b\) modulo \(H\)
    and simply write \(a \equiv b \pmod{H}\).
\end{definition}

\begin{theorem}
    Let \(H\) be a subgroup of a group \(G\). Then left (resp. right) congruence
    modulo \(H\) is an equivalence relation on \(G\).
\end{theorem}

\begin{proof}
    We prove that left congruence is an equivalence relation. The proof that
    right congruence is an equivalence relation is similar.

    \begin{enumerate}
        \item \emph{Reflexivity:} For all \(a \in G\), we have \(a^{-1}a = e \in
        H\), so \(a \equiv_L a \pmod{H}\).
        \item \emph{Symmetry:} Suppose \(a \equiv_L b \pmod{H}\). Then \(a^{-1}b
        \in H\), so \((a^{-1}b)^{-1} = b^{-1}a \in H\), i.e., \(b \equiv_L a
        \pmod{H}\).
        \item \emph{Transitivity:} Suppose \(a \equiv_L b \pmod{H}\) and \(b
        \equiv_L c \pmod{H}\). Then \(a^{-1}b \in H\) and \(b^{-1}c \in H\), so
        \((a^{-1}b)(b^{-1}c) = a^{-1}c \in H\), i.e., \(a \equiv_L c \pmod{H}\).
    \end{enumerate}
\end{proof}

\begin{theorem}
    \label{thm:cosets-equiv-classes}
    The equivalence class of an element \(a \in G\) under left (resp. right)
    congruence modulo \(H\) is given by \(aH = \{ah : h \in H\}\) (resp. \(Ha =
    \{ha : h \in H\}\)).
\end{theorem}

\begin{proof}
    The equivalence class of \(a\) under left congruence modulo \(H\) is given
    by
    \[
        \{ x \in G : x \equiv_L a \pmod{H} \} = \{ x \in G : x^{-1}a \in H \}.
    \]
    Since \(H < G\), \((x^{-1}a)^{-1} = a^{-1}x \in H\). Thus \(a^{-1}x = h\)
    for some \(h \in H\), and so \(x = ah\). The proof for right congruence is
    similar.
\end{proof}

\begin{definition}
    The sets \(aH\) and \(Ha\) in Theorem~\ref{thm:cosets-equiv-classes} are
    called the \emph{left coset} and \emph{right coset} of \(H\) containing
    \(a\), respectively.
\end{definition}

\begin{theorem}
    Let \(H\) be a subgroup of a group \(G\). Then \(\order{aH} = \order{H} =
    \order{Ha}\) for all \(a \in G\).
\end{theorem}

\begin{proof}
    Consider the map \(\phi: H \to aH\) defined by \(\phi(h) = ah\). We claim
    that \(\phi\) is a bijection. The map is clearly surjective. Suppose
    \(\phi(h) = \phi(h')\). Then \(ah = ah'\), so \(h = h'\). Thus \(\phi\) is
    injective. Therefore, \(\order{aH} = \order{H}\). The proof for
    \(\order{Ha}\) is similar.
\end{proof}

\begin{theorem}
    Let \(H\) be a subgroup of a group \(G\). Then \(G\) is the disjoint union
    of the left (resp. right) cosets of \(H\).
\end{theorem}

\begin{proof}
    Since left (resp. right) congruence modulo \(H\) is an equivalence relation
    on \(G\), the left (resp. right) cosets of \(H\) form a partition of \(G\).
\end{proof}

\begin{theorem}
    Let \(H < G\) and let \(a, b \in G\). If \(aH = bH\), then \(a^{-1}b \in
    H\).
\end{theorem}

\begin{proof}
    Since \(aH = bH\), we have \(a \in bH\), i.e., \(a = bh\) for some \(h \in
    H\). This implies that \(a^{-1}b = h \in H\).
\end{proof}

\begin{definition}
    \label{def:normal-subgroup}
    A subgroup \(N\) of a group \(G\) is said to be \emph{normal} if for all \(g
    \in G\) and \(n \in N\),
    \[
        gng^{-1} \in N.
    \]
    If \(N\) is a normal subgroup of \(G\), we write \(N \triangleleft G\).
\end{definition}

\begin{remark}
    The operation \(gng^{-1}\) is called \emph{conjugation} by \(g\). Thus,
    Definition~\ref{def:normal-subgroup} can be stated as: a subgroup \(N\) of a
    group \(G\) is normal if it is invariant under conjugation by elements of
    \(G\).
\end{remark}

\begin{example}
    It immediately follows from the definition that every subgroup of an abelian
    group is normal. In general, however, not all subgroups are normal. An
    example of a commutative group where every subgroup is normal is the
    quaternion group \(Q_8\).
\end{example}

\begin{theorem}
    A subgroup \(N\) of a group \(G\) is normal if and only if the left (resp.
    right) cosets of \(N\) are the same as the right (resp. left) cosets of
    \(N\). That is, left and right congruence modulo \(N\) define the same
    equivalence relation on \(G\).
\end{theorem}

\begin{proof}
    Suppose \(N \triangleleft G\), and let \(a \in G\). Let \(x \in gN\). Then
    \(x = gn\) for some \(n \in N\). Since \(N \triangleleft G\), we have
    \(gng^{-1} \in N\), so \(x = gn = gng^{-1}g = n'g\) for some \(n' \in N\).
    Thus \(x \in Ng\) and \(gN \subset Ng\). The proof that \(Ng \subset gN\) is
    similar. Conversely, suppose that left and right cosets of \(N\) coincide.
    Let \(g \in G\) and \(n \in N\). Then \(gn \in Ng = gN\), so \(gn = n'g\)
    for some \(n' \in N\). This implies that \(gng^{-1} = n' \in N\), and since
    \(g\) and \(n\) were arbitrary, we have \(N \triangleleft G\).
\end{proof}

\begin{theorem}
    \label{thm:normal-subgroup-props}
    Let \(N\) and \(K\) be subgroups of a group \(G\), with \(N\) normal in
    \(G\). Then:
    \begin{enumerate}[label=(\alph*)]
        \item \(N \cap K\) is a normal subgroup of \(K\);
        \item \(NK\) is a subgroup of G and \(N\) is normal in \(NK = \{nk : n
        \in N, k \in K\}\); and
        \item if \(K\) is normal in \(G\) and \(N \cap K = \{e\}\), then \(kn =
        nk\) for all \(n \in N\) and \(k \in K\).
    \end{enumerate}
\end{theorem}

\begin{proof}
    We prove each part in turn.
    \begin{enumerate}[label=(\alph*), wide]
        \item The intersection \(H \cap K\) is a subgroup of \(K\) by
        Theorem~\ref{thm:intersection-subgroups}. Since \(N \triangleleft G\),
        we must have \(gng^{-1} \in N\) for all \(g \in G\) and \(n \in N\). Let
        \(x \in N \cap K\) and \(k \in K\). Then \(x \in K\) and because \(K <
        G\) (i.e., closed), we must have \(kxk^{-1} \in K\). On the other hand,
        we also have \(x \in N\) and because \(k \in K \subset G\) we have
        \(kxk^{-1} \in N\). Thus \(kxk^{-1} \in N \cap K\), and so \(N \cap K
        \triangleleft K\).
        
        \item Let \(a, b \in NK\). Then \(a = n_1k_1\) and \(b = n_2k_2\) for
        some \(n_1, n_2 \in N\) and \(k_1, k_2 \in K\). Since \(N \triangleleft
        G\), and \(k^{-1}_1k_1 = e \in N\), we have
        \[
            ab = (n_1k_1)(n_2k_2) = n_1k_1n_2(k^{-1}_1k_1)k_2 = n_1(k_1n_2k^{-1}_1)k_1k_2
        \]
        Since \(N\) is normal \(k_1n_2k^{-1}_1\) must be equal to some element
        \(n_3 \in N\), so \(ab = n_1n_3k_1k_2 \in NK\).
    \end{enumerate}
\end{proof}

\begin{theorem}
    \label{thm:quotient-group}
    Let \(N \triangleleft G\). The set of all (left) cosets of \(N\) in \(G\)
    forms a group under the operation
    \[
        (aN)(bN) = abN,
    \]
    for all \(a, b \in G\). This is called the \emph{quotient group} of \(G\)
    modulo \(N\), denoted by \(G/N\).
\end{theorem}

\begin{proof}
    First we show that the operation is well-defined. Suppose \(a, a', b, b' \in
    G\) such that \(aN = a'N\) and \(bN = b'N\). Then \(a^{-1}a' \in N\) and
    \(b^{-1}b' \in N\). The first statement implies that \(b^{-1}a^{-1}a' \in
    b^{-1}H = Hb^{-1}\) (because \(N\)) is normal. Thus there exists an element
    \(n \in N\) such that \(b^{-1}a^{-1}a' = nb^{-1}\); multiplying on the right
    by \(b'\) gives us \(b^{-1}a^{-1}a'b' = nb^{-1}b' \in N\). We can simplify
    this to \((ab)^{-1}(a'b') \in N\), which implies that \((ab)N = (a'b')N\).

    The operation is associative because multiplication in \(G\) is associative.
    The identity element of the group is \(eN = N\) where \(e\) is the identity
    element of \(G\). The inverse of \(aN\) is \(a^{-1}N\). Thus the set of all
    left cosets of \(N\) in \(G\) forms a group under the operation \((aN)(bN) =
    abN\).
\end{proof}

\begin{remark}
    The definition of a quotient group in Theorem~\ref{thm:quotient-group} uses
    left cosets for concreteness, but since we are dealing with a normal
    subgroup, it holds for right cosets as well.
\end{remark}

\begin{theorem}
    Let \(\phi : G \to H\) be a group homomorphism. Then the \(\ker \phi\) is a
    normal subgroup of \(G\). Conversely, if \(N\) is a normal subgroup of
    \(G\), then the map
    \[
        \pi : G \to G/N
    \]
    defined by \(\pi(g) = gN\) is an epimorphism with kernel \(N\). We call
    \(\pi\) the \emph{canonical projection}.
\end{theorem}

\begin{proof}
    We have already established that \(\ker \phi < G\). Let \(g \in G\) and \(n
    \in \ker \phi\). Write \(e'\) for the identity element of \(H\). Then
    \[
        \phi(gng^{-1}) = \phi(g)\phi(n)\phi(g^{-1}) = \phi(g)e'\phi(g^{-1}) = \phi(g)\phi(g^{-1}) = e'.
    \]
    Therefore, \(gng^{-1} \in \ker \phi\).

    On the other hand, since each element of \(G/N\) is of the form \(gN\) for
    some \(g \in G\), the map \(\pi\) is surjective. By our definition of coset
    multiplication, we also have
    \[
        \pi(g)\pi(g') = gN \cdot g'N = gg'N = \pi(gg'),
    \]
    so that \(\pi\) is a group homomorphism. Finally, the kernel of \(\pi\) is
    \[
        \ker \pi = \{g \in G : gN = N\} = N.
    \]
\end{proof}

\begin{theorem}
    \label{thm:induced-homomorphism}
    Let \(N\) be a normal subgroup of a group \(G\). If \(\phi: G \to H\) is a
    group homomorphism such that \(N \subset \ker \phi\), then there exists a
    unique group homomorphism \(\overline{\phi}: G/N \to H\) so that the diagram
    \[
        \begin{tikzcd}
            G \arrow{r}{\phi} \arrow[swap]{d}{\pi} & H \\
            G/N \arrow[dashed]{ur}{\overline{\phi}}
        \end{tikzcd}
    \]
    commutes. Here \(\pi: G \to G/N\) is the canonical projection map, defined
    by \(\pi(g) = gN\).
\end{theorem}

\begin{proof}
    Let \(b \in aN\) for some \(a \in G\). Then \(b = an\) for some \(n \in N\).
    Thus \(\phi(b) = \phi(an) = \phi(a)\phi(n)\). Since \(N \subset \ker \phi\),
    we have \(\phi(n) = e\), so \(\phi(b) = \phi(a)\). This implies that
    \(\phi\) sends elements of \(G/N\) to elements of \(H\) in a well-defined
    manner and the map \(\overline{\phi}\) defined by \(\overline{\phi}(aN) =
    \phi(a)\) is similarly well-defined. Since
    \[
    \overline{\phi}(aNbN) = \overline{\phi}(abN) = \phi(ab) = \phi(a)\phi(b) = \overline{\phi}(aN)\overline{\phi}(bN),
    \]
    the map \(\overline{\phi}\) is a group homomorphism. Because
    \(\overline{\phi}\) is determined by \(\phi\), the map \(\overline{\phi}\)
    is unique.

    We finally have
    \[
        (\overline{\phi} \circ \pi)(g) = \overline{\phi}(\pi(g)) = \overline{\phi}(gN) = \phi(g),
    \]
    and the diagram therefore commutes.
\end{proof}

\begin{remark}
    The map \(\overline{\phi}\) is called the \emph{homomorphism induced by
    \(\phi\)}.
\end{remark}

\begin{theorem}
    Given the same conditions as in the preceding theorem, the following
    statements hold:
    \begin{enumerate}[label=(\alph*)]
        \item \(\img \phi = \img \overline{\phi}\).
        \item \(\ker \overline{\phi} = (\ker \phi)/N\).
        \item The map \(\overline{\phi}\) is an isomorphism if and only if
        \(\phi\) is an epimorphism and \(\ker \phi = N\).
    \end{enumerate}
\end{theorem}

\begin{proof}
    We prove each part in turn.

    \begin{enumerate}[label=(\alph*), wide]
        \item For any \(x \in \img \phi\), there exists an element \(g \in G\)
        such that \(x = \phi(g)\). Then \(\overline{\phi}(gN) = \phi(g) = x\),
        so \(x \in \img \overline{\phi}\). Conversely if \(x \in \img
        \overline{\phi}\), then there exists some coset \(gN\) such that \(x =
        \overline{\phi}(gN) = \phi(g)\), so \(x \in \img \phi\). Thus \(\img
        \phi = \img \overline{\phi}\).
        
        \item If a coset \(gN \in \ker \overline{\phi}\), then
        \(\overline{\phi}(gN) = \phi(g) = e\), so \(g \in \ker \phi\). This
        implies that \(\ker \overline{\phi} = \{gN : g \in \ker \phi\} = (\ker
        \phi)/N\).
        
        \item If \(\overline{\phi}\) is an isomorphism, then it is also an
        epimorphism. Since \(\img \phi = \img \overline{\phi}\),it follows that
        \(\phi\) is also an epimorphism. Since \(\overline{\phi}\) is also
        injective, we have \(\ker \overline{\phi} = \{e_{G/N}\} = N\) (by our
        definition of the identity element of \(G/N\)). Thus \(\ker \phi = N\).
        
        Conversely, if \(\phi\) is an epimorphism and \(\ker \phi = N\), then
        \(\img \phi = H = \img \overline{\phi}\) and \(\overline{\phi}\) is an
        epimorphism. Since \(\ker \overline{\phi} = (\ker \phi)/N = \{e_{G/N}\}
        = N\), \(\overline{\phi}\) is also injective and therefore an
        isomorphism.
    \end{enumerate}
\end{proof}

\begin{example}
    We consider one important class of examples. The cyclic group \(\Z/n\Z\)
    have been previously defined as the equivalence classes of integers under
    the relation
    \[
        a \equiv b \pmod{n} \quad \text{if and only if} \quad n \mid (a - b).
    \]
    The condition \(n \mid (a - b)\) is equivalent to \(a - b \in n\Z\) (in
    additive notation). Since \(\Z\) is abelian, \(n\Z\) is a normal subgroup of
    \(\Z\). The quotient group \(\Z/n\Z\) is then the set of all cosets of
    \(n\Z\) in \(\Z\), i.e., the set of all equivalence classes of integers
    modulo \(n\). Using additive notation, we can write the equivalence class of
    an integer \(a\) as
    \[
        [a]_n = \{a + kn : k \in \Z\} = a + n\Z.
    \]
    Addition in \(\Z/n\Z\) matches our definition of coset multiplication above.

    Now consider some map \(\epsilon_g: \Z \to G\) (for some group \(G\) with
    order \(n\)) that sends an integer \(\nu\) to the element \(g^\nu \in G\).
    This map is a group homomorphism by the properties of exponents. The kernel
    of \(\epsilon_g\) is then given by
    \[
        \ker \epsilon_g = \{ \nu \in \Z : g^\nu = e \},
    \]
    which are precisely the integers \(\nu\) that divide the order of \(g\),
    which is \(n\). This in turn is the subgroup \(n\Z\). By Theorem
    \ref{thm:induced-homomorphism}, \(\epsilon_g\) can be decomposed through the
    quotient group \(\Z/n\Z\):
    \[
        \begin{tikzcd}
            \Z \arrow{r}{\epsilon_g} \arrow[swap]{d}{\pi} & G \\
            \Z/n\Z \arrow[dashed, "\overline{\epsilon}_g" below]{ur}
        \end{tikzcd}
    \]
    That is, the map \(\epsilon_g\) induces a unique homomorphism
    \(\overline{\epsilon}_g: \Z/n\Z \to G\) such that \(\epsilon_g =
    \overline{\epsilon}_g \circ \pi\) where again \(\pi: \Z \to \Z/n\Z\) is the
    canonical projection map that sends an integer to its equivalence class
    modulo \(n\). The map \(\overline{\epsilon}_g\) is an isomorphism since
    \(\ker \epsilon_g = n\Z\) and \(\epsilon_g\) is an epimorphism.
\end{example}

\begin{theorem}[Canonical decomposition of a group homomorphism]
    Every group homomorphism \(\phi: G \to H\) can be decomposed as follows:
    \[
        \begin{tikzcd}
            G \arrow[r, two heads] \arrow[rrr, bend left, "\phi"]   & G/\ker\phi \arrow[r,"\sim" above, "\overline{\phi}" below]   & \operatorname{im}\phi \arrow[r, hook]  & H
        \end{tikzcd}
    \]
\end{theorem}

\begin{proof}
    We omit the proof as it is simply a restatement of everything we have
    established so far.
\end{proof}

The next result is immediately implied in the preceding theorem but we state it
nonetheless because of its importance.

\begin{theorem}[First isomorphism theorem]
    Let \(\phi: G \to H\) be a group homomorphism. Then
    \[
        G/\ker \phi \cong \img \phi.
    \]
\end{theorem}

\begin{theorem}[Second isomorphism theorem]
    Let \(K\) and \(N\) be subgroups of a group \(G\) such that
    \(\normal{N}{G}\). Then \(N \cap K\) is normal in \(K\), and
    \[
        K/(N \cap K) \cong NK/N.
    \]
\end{theorem}

\begin{proof}
    Recall from Theorem~\ref{thm:normal-subgroup-props} that \(NK\) is a
    subgroup of \(G\) and \(N\) is normal in \(NK\). Consider the map
    \[
        \phi : K \to NK/N.
    \]
    Every element of \(NK/N\) is of the form \(nkN\) for some \(n \in N\) and
    \(k \in K\). Since \(N \triangleleft G\), this implies that \(nk = kn'\) for
    some \(n' \in N\) and \(nkN = kn'N = kN\). That is, every element of
    \(NK/N\) is the image of some element of \(K\) and \(\phi\) is surjective.
    Moreover, \(\phi\) is the composition
    \[
        \begin{tikzcd}
            K \arrow[r, "\iota", hook] & NK \arrow[r, "\pi", two heads] & NK/N.
            \end{tikzcd}
    \]
    By the first isomorphism theorem, we then have
    \[
        K/\ker \phi \cong NK/N.
    \]
    But then
    \begin{align*}
        \ker \phi &= \{k \in K : \phi(k) = N\} = \{k \in K : kN = N\} = \{k \in K : k \in N\}\\ &= N \cap K,
    \end{align*}
    and the result follows.
\end{proof}

\section{Group actions}

\begin{definition}
    The action of a group \(G\) on an object \(X\) in a category \(\catg{C}\) is
    a (group) homomorphism
    \[
        \alpha: G \to \Aut_{\catg{C}}(X),
    \]
\end{definition}

\begin{definition}
    An (left) action of a group \(G\) on a set \(X\) is a map
    \[
        \alpha: G \times X \to X
    \]
    such that for all \(g, h \in G\) and \(x \in X\), we have
    \begin{enumerate}[label=(\alph*)]
        \item \(\alpha(e, x) = x\) and
        \item \(\alpha(gh, x) = \alpha(g, \alpha(h, x)).\)
    \end{enumerate}

    As with multiplication in a group, it would be convenient to simply write
    \(\alpha(g, x)\) as \(gx\).
\end{definition}

\begin{example}
    Every group \(G\) acts on the underlying set of the group through the map
    \(\alpha: G \times G \to G\) defined by \(\alpha(g, x) = gx\) (where the
    multiplication on the right-hand side is the group operation of \(G\)). This
    action is referred to as \emph{left multiplication}.

    We can also verify that the map \(\gamma: G \times G \to G\) defined by
    \[
        \gamma(g, x) = gxg^{-1}
    \]
    is an action of \(G\) on itself. Indeed, we have
    \[
        \gamma(e, x) = exe^{-1} = x
    \]
    and
    \[
        \gamma(g, \gamma(h, x)) = g(hxh^{-1})g^{-1} = (gh)x(gh)^{-1} = \gamma(gh, x),
    \]
    for all \(g, h, x \in G\). This action is called \emph{conjugation}.
\end{example}

\begin{sectionthm}
    An action of a group \(G\) on a set \(X\) is said to be \emph{faithful} if
    for all \(g \in G\), \(gx = x\) if and only if \(g = e\).
\end{sectionthm}

\section{The Sylow theorems}