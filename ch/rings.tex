\chapter{Rings and modules}
\label{ch:rings}

\section{Rings and homomorphisms}

\begin{definition}
    A \emph{ring} is a triple \((R, +, \cdot)\), where \(R\) is a nonempty set
    and \(+\) and \(\cdot\) are two binary operations (often referred to as
    `addition' and `multiplication', respectively) such that:
    \begin{enumerate}[label=(\alph*)]
        \item \((R, +)\) is an abelian group.
        \item Multiplication is associative, i.e. \((a \cdot b) \cdot c = a
        \cdot (b \cdot c)\) for all \(a, b, c \in R\).
        \item Multiplication is distributive over addition, i.e. \(a \cdot (b +
        c) = a \cdot b + a \cdot c\) and \((a + b) \cdot c = a \cdot c + b \cdot
        c\) for all \(a, b, c \in R\).
        \item There exists a distinguished element \(1 \in R\) such that \(1
        \cdot a = a \cdot 1 = a\) for all \(a \in R\).
    \end{enumerate}
\end{definition}

More succinctly, axioms (b) and (d) above tell us that \((R, \cdot)\) is a
monoid with identity element \(1\).

Some texts may refer to the triple \((R, +, \cdot)\) satisfying the above axioms
as a \emph{ring with unity} or a \emph{ring with \(1\)} and only require that
the first three axioms be satisfied for \((R, +, \cdot)\) to be a ring. Rings
without unity are sometimes referred to as \emph{rngs} in the literature. For
example, the set of even integers \(2\Z\) with the usual operations of addition
and multiplication satisfies all the axioms above except for the requirement
that the multiplicative identity \(1\) be in \(2\Z\). Thus it is a `rng' by the
preceding definition. Nevertheless, we shall follow the above definition in this
text and all rings will be assumed to have a multiplicative identity unless
otherwise stated. If multiplication is commutative, i.e. \(a \cdot b = b \cdot
a\) for all \(a, b \in R\), then we say that \(R\) is a \emph{commutative ring}.
As in Chapter~\ref{ch:groups}, we will often write \(ab\) instead of \(a \cdot
b\).

Some elementary properties of rings derive from the properties of additive
groups and monoids. The identity element of the abelian group \((R, +)\) is
often denoted by \(0\), and is called the \emph{zero} of the ring \(R\). Since
\(R\) is an abelian group under addition, it follows that \(0\) must be unique,
as is the additive inverse \(-a\) of every element \(a \in R\). Moreover, for
all \(a, b \in R\), we have
\[
    -(-a) = a, \quad -0 = 0, \quad \text{and} \quad -(a + b) = (-a) + (-b).
\]
The usual laws for multiples of the form \(na\) where \(a \in R\) and \(n\) is
an integer also hold in rings. To wit, we have
\[
    (m+n)a = ma + na, \quad (-m)a = -(ma), \quad \text{and} \quad m(na) = (mn)a
\]
and because \(R\) is also abelian we have
\[
    m(a+b) = ma + mb
\]
for all \(a, b \in R\) and \(m, n \in \mathbb{Z}\). Because \(R\) is also a
monoid under multiplication, the distinguished element \(1 \in R\) (called the
\emph{unity}) must likewise be unique and the usual laws of exponents must
similarly hold, so that we have, for all \(a \in R\) and \(m, n \in
\mathbb{Z}\),
\[
    a^m a^n = a^{m+n} \quad \text{and} \quad (a^m)^n = a^{mn}.
\]

Other properties do not directly derive from \(R\) being an abelian group or a
monoid under addition and multiplication, respectively, but are consequences of
the distributive law. In particular we have the following results.

\begin{theorem}
    Let \(R\) be a ring. Then
    \begin{enumerate}[label={\normalfont(\alph*)}]
        \item \(0a = a0 = 0\) for all \(a \in R\);
        \item if \(R\) is not the zero ring, then \(0 \neq 1\);
        \item \((-a)b = a(-b) = -(ab)\) for all \(a, b \in R\);
        \item For all \(a, b \in R\),
        \[
            \left(\sum_{i=1}^{m} a_i\right)\left(\sum_{j=1}^{n} b_j\right) = \sum_{i=1}^{m} \sum_{j=1}^{n} a_i b_j;
        \]
        \item \((na)b = n(ab) = a(nb)\) for all \(a, b \in R\) and \(n \in
        \mathbb{Z}\).
    \end{enumerate}
\end{theorem}

\begin{proof}\(\)
    \begin{enumerate}[label=(\alph*), wide]
        \item Since \(0\) is the additive identity, we have \(0a = (0 + 0)a = 0a
        + 0a\), whence \(0a = 0\). Similarly, \(a0 = a(0 + 0) = a0 + a0\), so
        \(a0 = 0\).

        \item If \(0 = 1\), then for all \(a \in R\), we have \(a = a1 = a0 =
        0\), from which it follows that \(R\) contains only a single element
        which is both the additive and multiplicative identity. We call this the
        \emph{zero ring}. As in the case of groups, we can consider any
        singleton \(\{*\}\) with addition and multiplication defined by \(* + *
        = *\) and \(* \cdot * = *\), respectively, to be the zero ring
        (analogous to the trivial group).
        
        \item From (a) above and the distributive law, we have
        \[
            0 = 0b = (a + (-a))b = ab + (-a)b,
        \]
        whence \((-a)b = -(ab)\). An analogous argument shows that \(a(-b) =
        -(ab)\).

        Since \(-(-a) = a\), we then have, as an immediate consequence, that
        \((-a)(-b) = a(-(-b)) = a(b) = ab\) and in particular \((-1)(-1) = 1\)
        and \((-1)a = -a\).

        \item For \(m = 1\), the proof is by induction on \(n\), with the case
        \(n = 2\) being left distribution. Having established this rule for \(m
        = 1\), a second induction on \(m\) completes the proof.
        
        \item For positive \(n\), the first equation is given by \(ab + \cdots +
        ab = (a + \cdots + a)b\) and is thus a special case of (d) above. For
        negative \(n\), we can use the result in (c).
    \end{enumerate}
\end{proof}

We now look at some examples of rings and introduce some terminology.

\begin{example}[Integral domains]
    \label{ex:integral-domains}
    Rings generalize our notion of number systems. In particular, the familiar
    sets \(\Z\), \(\Q\), \(\R\) and \(\C\) are all commutative rings under the
    usual operations of addition and multiplication. 

    The additive group \(\Z/n\Z\) of integers modulo \(n\) is a ring under the
    usual addition modulo \(n\) and multiplication modulo \(n\) which we define
    as \([a] \cdot [b] = [ab]\). Observe however that while additive
    cancellation holds in both \(\Z\) and \(\Z/n\Z\), multiplicative
    cancellation does not hold in \(\Z/n\Z\). For example, in \(\Z/6\Z\), we
    have \([2] \cdot [3] = [0] = [2] \cdot [0]\) but \([3] \neq [0]\). This is
    because the element \([3]\) is a zero divisor in \(\Z/6\Z\), i.e., there
    exists a nonzero element \([a] \in \Z/6\Z\) such that \([3] \cdot [a] =
    [0]\). In general, an element \(a \in R\) is a left (resp. right) \emph{zero
    divisor} if there exists a nonzero element \(b \in R\) such that \(ab = 0\)
    (resp. \(ba = 0\)). An element that is both a left and right zero divisor is
    simply called a (two-sided) \emph{zero divisor}. In any nonzero ring \(R\),
    the zero element is trivially a zero divisor. We say that a (left, right,
    two-sided) zero divisor is \emph{nontrivial} if it is not the zero element
    of \(R\).

    A nonzero commutative ring \(R\) is called an \emph{integral domain} if it
    has no nontrivial zero divisors. In other words, \(R\) is an integral domain
    if for all \(a, b \in R\), \(ab = 0\) implies that either \(a = 0\) or \(b =
    0\). This condition guarantees that multiplicative cancellation also holds
    in \(R\). Indeed if \(ab = ac\) for some \(a \neq 0\) in an integral domain
    \(R\), then \(ab - ac = a(b - c) = 0\); but since \(a \neq 0\) and \(R\) is
    an integral domain, we must have \(b - c = 0\) and thus \(b = c\). The
    familiar sets \(\Z\), \(\Q\), \(\R\) and \(\C\) are all integral domains but
    \(\Z/n\Z\) is an integral domain if and only if \(n\) is a prime number.
    Indeed if \(n = ab\) for some integers \(a, b\) with \(1 < a, b < n\), then
    \([a] \cdot [b] = [ab] = [n] = [0]\) and \([a] \neq [0]\) and \([b] \neq
    [0]\), so \(\Z/n\Z\) is not an integral domain. On the other hand if \(n =
    p\) is a prime number, so that for all integers \(a, b\) with \(1 < a, b <
    p\), \([ab] = [0]\) implies that \([a] = [0]\) or \([b] = [0]\), and thus
    \(\Z/p\Z\) is an integral domain.
\end{example}

\begin{example}[Units, division rings, fields]
    An element \(u\) of a ring \(R\) is a left (resp. right) \emph{unit}
\end{example}

\begin{example}[Polynomial rings]
    Let \(R\) be a ring. A polynomial \(p\) in an indeterminate \(x\) with
    coefficients in \(R\) is an expression of the form
    \begin{equation}
        \label{eq:polynomial}
        p(x) = \sum_{i \geq 0} a_i x^i = a_0 + a_1 x + a_2 x^2 + \cdots
    \end{equation}
    where \(a_i \in R\) for all \(i \geq 0\) and \(a_i = 0\) for all but
    finitely many \(i\). We say that two polynomials \(p(x) = \sum a_i x^i\) and
    \(q(x) = \sum b_i x^i\) are equal if and only if their coefficients are
    equal, i.e., \(a_i = b_i\) for all \(i \geq 0\). We shall denote the set of
    all polynomials in the indeterminate \(x\) with coefficients in \(R\) by
    \(R[x]\). Now since all but finitely many of the coefficients of a
    polynomial are zero, we can alternatively write \eqref{eq:polynomial} as
    \[
        p(x) = a_0 + a_1 x + a_2 x^2 + \cdots + a_n x^n
    \]
    where \(n\) is the largest integer such that \(a_n \neq 0\). We call \(n\)
    the \emph{degree} of the polynomial \(p(x)\) and write \(\deg p = n\). The
    zero polynomial is the polynomial with all coefficients equal to zero, and
    is denoted by \(0\). The constant polynomial \(a_0\) (for any \(a_0 \in R\))
    is the polynomial with all coefficients equal to zero except for the
    constant term, which is equal to \(a_0\). The polynomial \(1\) is the
    constant polynomial with constant term equal to \(1\). The sum of two
    polynomials is defined by adding the coefficients of like terms, while the
    product of two polynomials is defined by the distributive law. That is, for
    any two polynomials \(p(x) = \sum a_i x^i\) and \(q(x) = \sum b_i x^i\) in
    \(R[x]\), we define their sum \(p(x) + q(x)\) as
    \begin{align*}
        p(x) + q(x) & = \sum (a_i + b_i) x^i = (a_0 + b_0) + (a_1 + b_1)x + \cdots\\
        &= (a_0 + a_1 x + \cdots) + (b_0 + b_1 x + \cdots) = \sum a_i x^i + \sum b_i x^i
    \end{align*}
    and their product \(p(x)q(x)\) as
    \begin{align*}
        p(x)q(x) & = \left(\sum a_i x^i\right)\left(\sum b_i x^i\right) = \sum \left(\sum_{j=0}^{i} a_j b_{i-j}\right) x^i\\
        &= a_0 b_0 + (a_0 b_1 + a_1 b_0)x + (a_0 b_2 + a_1 b_1 + a_2 b_0)x^2 + \cdots.
    \end{align*}
    With the definitions given above, we can see that \(R[x]\) is a ring under
    the operations of polynomial addition and multiplication with the zero
    polynomial as the additive identity and the constant polynomial \(1\) as the
    multiplicative identity. The ring \(R[x]\) is called the \emph{ring of
    polynomials over \(R\)}.
\end{example}


\begin{definition}
    \label{def:ring-homomorphism}
    A \emph{homomorphism of rings} is a (set) function \(\phi: R \to S\) between
    two rings \(R\) and \(S\) such that for all \(a, b \in R\),
    \begin{enumerate}[label=(\alph*)]
        \item \(\phi(a + b) = \phi(a) + \phi(b)\);
        \item \(\phi(ab) = \phi(a)\phi(b)\);
        \item \(\phi(1) = 1'\).
    \end{enumerate}
    Here the addition and multiplication on the left-hand side are those of
    \(R\), while the addition and multiplication on the right-hand side are
    those of \(S\); the unity of \(R\) is denoted by \(1\) and the unity of
    \(S\) is denoted by \(1'\).
\end{definition}

From the definition it then follows that \(\phi\) is a homomorphism of the
additive groups \(R\) and \(S\) so that \(\phi(0) = 0\) and \(\phi(-a) =
-\phi(a)\) for all \(a \in R\). However, as part of our definition of a ring
homomorphism, we require that \(\phi(1) = 1'\) so that \(\phi\) is also a
homomorphism of the monoids \((R, \cdot)\) and \((S, \cdot)\). Indeed the
condition \(\phi(1) = 1'\) does not directly follow from (a) and (b) in
Definition~\ref{def:ring-homomorphism}. For example, consider a commutative ring
\(R\) that contains an idempoent element \(e\), i.e., an element \(e\)
satisfying \(e^2 = e\). Then the function \(\phi: R \to R\) defined by \(x
\mapsto xe\) satisfies conditions (a) and (b) above, viz., \[\phi(x + y) = (x +
y)e = xe + ye = \phi(x) + \phi(y);\] we also have \(\phi(x) \phi(y) = (xe)(ye) =
(xy)e\) because \(R\) is commutative and \(e\) is idempotent and thus \(\phi(xy)
= \phi(x) \phi(y)\). However, \(\phi(1) = 1e = e\) and thus condition (c) is not
satisfied.

For any ring \(R\), the identity map \(\id_R: R \to R\) is a ring homomorphism.
Moreover, given any two ring homomorphisms \(\phi: R \to S\) and \(\psi: S \to
T\), their composition \(\psi \circ \phi: R \to T\) is also a ring homomorphism.
Thus the collection of all rings and ring homomorphisms forms a category, which
we denote by \(\Ring\). Injective, surjective and bijective ring homomorphisms
(in the sense of set functions) are likewise the monomorphisms, epimorphisms and
isomorphisms in the category \(\Ring\), respectively.

As in the case of the trivial group in \(\Grp\), the zero ring is terminal in
\(\Ring\). That is, the only ring homomorphism from any ring \(R\) to the zero
ring is the trivial homomorphism that sends every element of \(R\) to the zero
element of the zero ring. However, note that the zero ring is not an initial
object in \(\Ring\): indeed given a map \(\phi: 0 \to R\) from the zero ring to
any ring \(R\), the condition \(\phi(1) = 1\) is satisfied if and only if \(R\)
is itself the zero ring. So does \(\Ring\) then have an initial object?

Given any ring \(R\), we can define the group homomorphism \(\phi: \mathbb{Z}
\to R\) by \(\phi(n) = n1_R\) for all \(n \in \mathbb{Z}\). (This is the
exponential map given as in Example~\ref{ex:exponential-maps} of
Chapter~\ref{ch:groups}.) But observe that this is also a ring homomorphism,
since \(\phi(1) = 1 \cdot 1_R = 1_R\) and
\[
    \phi(mn) = (mn)1_R = m(n1_R) =^* (m1_R)\cdot(n1_R) = \phi(m)\phi(n)
\]
with the starred equality following from the distributive law. Since \(\phi\) is
determined by the condition that \(\phi(1) = 1_R\) and that \(\phi\) preserve
addition, it must be unique. Thus we have shown that \(\mathbb{Z}\) is initial
in \(\Ring\).

Ring homomorphisms also preserve units: that is, if \(u\) i

\section{Ideals and quotient rings}